\apendice{Especificación de diseño}

\section{Introducción}

\section{Diseño de datos}


\subsection{Diccionario de datos}

Se adjunta a continuación, para cada tabla, el diccionario de datos correspondiente.

\subsubsection{Tabla de usuarios (\texttt{Users})}
Esta tabla (\ref{datadic:users}) almacena la información relevante de cada usuario registrado en la base de datos. Si se comprueba el diagrama E-R, se puede comprobar que una peculiaridad de esta tabla es que es una ISA. Debido a que, además del discriminante (usuario administrador o estándar), únicamente hay un atributo que diferencia ambas clases, se ha decidido implementar utilizando una sola tabla.

\begin{table}
	\small
	\begin{centering}
		\begin{tabular}{@{}p{6em} p{6em} p{6em} p{12em}@{}}
			\toprule
			\textbf{Nombre} & \textbf{Tipo y tamaño} & \textbf{Otros} & \textbf{Descripción}\\
			\midrule
			\textbf{\underline{id}} & &  \textbf{\underline{PK}} & Identificador \\
			username &  &  &   Nombre de usuario\\
			email &  &  &  Email del usuario \\
			password &  &  &  Contraseña\\
			user\_first\_name &  &  &  Nombre \\
			user\_last\_name &  &  &  Apellidos \\
			user\_rol &  & Por defecto: <<standard>> & Rol. Puede ser <<standard>> o <<admin>>. \\
			n\_urls\_accepted &  &  &  Atributo de usuarios <<standard>>. Contador de aquellas instancias que han sido reportadas y aceptadas por un administrador. \\
			\bottomrule
		\end{tabular}
	\end{centering}
	\caption[Diccionario de datos: Users]{Diccionario de datos. Tabla correspondiente a la clase \texttt{Users}.}
	\label{datadic:users}
\end{table}

\subsubsection{Tabla de instancias (\texttt{Available\_instances})}

Esta tabla (\ref{datadic:instances}) almacena la información asociada a cada instancia disponible en el \textit{dataset}. Las instancias son URLs que pueden ser utilizadas posteriormente para entrenar y probar modelos. Además, esta tabla también almacena enlaces reportados por usuarios por pertenecer a una lista blanca o negra.

\begin{table}
	\small
	\begin{centering}
		\begin{tabular}{@{}p{6em} p{6em} p{6em} p{6em} p{6em}@{}}
			\toprule
			\textbf{Nombre} & \textbf{Tipo de dato y tamaño} & \textbf{Por defecto}& \textbf{Otros} & \textbf{Descripción}\\
			\midrule
			instance\_id & INTEGER 8 & anterior & PK & Identificador \\
			reviewed\_by &  &  &  &  \\
			instance\_url &  &  &  &   \\
			instance\_fv &  &  &  &   \\
			instance\_class &  &  &  &   \\
			colour\_list &  &  &  &   \\
			instance\_labels &  &  &  &   \\
			\bottomrule
		\end{tabular}
	\end{centering}
	\caption[Diccionario de datos: Available\_instances]{Diccionario de datos. Tabla correspondiente a la clase \texttt{Available\_instances}.}
	\label{datadic:instances}
\end{table}


\subsubsection{Tabla de modelos (\texttt{Available\_models})}

Esta tabla (\ref{datadic:models}) almacena los modelos de aprendizaje que han sido creados por un administrador y están disponibles en la \textit{web} para su uso.

\begin{table}
	\small
	\begin{centering}
		\begin{tabular}{@{}p{6em} p{6em} p{6em} p{6em} p{6em}@{}}
			\toprule
			\textbf{Nombre} & \textbf{Tipo de dato y tamaño} & \textbf{Por defecto}& \textbf{Otros} & \textbf{Descripción}\\
			\midrule
			model\_id & INTEGER 8 & anterior & PK & Identificador \\
			created\_by &  &  &  &  \\
			model\_name &  &  &  &   \\
			file\_name &  &  &  &   \\
			date &  &  &  &   \\
			default &  &  &  &   \\
			visible &  &  &  &   \\
			random\_state &  &  &  &   \\
			\bottomrule
		\end{tabular}
	\end{centering}
	\caption[Diccionario de datos: Available\_models]{Diccionario de datos: tabla correspondiente a la clase \texttt{Available\_models}.}
	\label{datadic:models}
\end{table}

Si se comprueba el diagrama E-R, se puede comprobar que esta tabla es una ISA. En este caso se ha decidido implementar considerando que es una interrelación 1:1. Por ello, se ha creado una tabla para la generalización y otra tabla para cada una de las especializaciones. Estas se enumeran a continuación:

\begin{itemize}
	\item \textbf{\texttt{Available\_co\_forests}}: los modelos pertenecientes a este algoritmo poseen los atributos de la tabla~\ref{datadic:coforest}.
	
	\begin{table}
		\small
		\begin{centering}
			\begin{tabular}{@{}p{6em} p{6em} p{6em} p{6em} p{6em}@{}}
				\toprule
				\textbf{Nombre} & \textbf{Tipo de dato y tamaño} & \textbf{Por defecto}& \textbf{Otros} & \textbf{Descripción}\\
				\midrule
				model\_id & INTEGER 8 & anterior & PK & Identificador \\
				max\_features &  &  &  &  \\
				thetha &  &  &  &   \\
				n\_trees &  &  &  &   \\
				\bottomrule
			\end{tabular}
		\end{centering}
		\caption[Diccionario de datos: Available\_co\_forests]{Diccionario de datos: tabla correspondiente a la clase \texttt{Available\_co\_forests}.}
		\label{datadic:coforest}
	\end{table}

	\item \textbf{\texttt{Available\_tri\_trainings}}: los modelos pertenecientes a este algoritmo poseen los atributos de la tabla~\ref{datadic:tritraining}.

	\begin{table}
		\small
		\begin{centering}
			\begin{tabular}{@{}p{6em} p{6em} p{6em} p{6em} p{6em}@{}}
				\toprule
				\textbf{Nombre} & \textbf{Tipo de dato y tamaño} & \textbf{Por defecto}& \textbf{Otros} & \textbf{Descripción}\\
				\midrule
				model\_id & INTEGER 8 & anterior & PK & Identificador \\
				cls\_one &  &  &  &  \\
				cls\_two &  &  &  &   \\
				cls\_three &  &  &  &   \\
				\bottomrule
			\end{tabular}
		\end{centering}
		\caption[Diccionario de datos: Available\_tri\_trainings]{Diccionario de datos: tabla correspondiente a la clase \texttt{Available\_tri\_trainings}.}
		\label{datadic:tritraining}
	\end{table}

	\item \textbf{\texttt{Available\_democratic\_cos}}: los modelos pertenecientes a este algoritmo poseen los atributos de la tabla~\ref{datadic:democraticco}.

	\begin{table}
		\small
		\begin{centering}
			\begin{tabular}{@{}p{6em} p{6em} p{6em} p{6em} p{6em}@{}}
				\toprule
				\textbf{Nombre} & \textbf{Tipo de dato y tamaño} & \textbf{Por defecto}& \textbf{Otros} & \textbf{Descripción}\\
				\midrule
				model\_id & INTEGER 8 & anterior & PK & Identificador \\
				base\_clss &  &  &  &  \\
				\bottomrule
			\end{tabular}
		\end{centering}
		\caption[Diccionario de datos: Available\_democratic\_cos]{Diccionario de datos: tabla correspondiente a la clase \texttt{Available\_democratic\_cos}.}
		\label{datadic:democraticco}
	\end{table}
\end{itemize}

\subsubsection{Tablas de relación}

Se enumeran a continuación las tablas que son resultado de una relación entre entidades.

\begin{itemize}
	\item \textbf{\texttt{Candidate instances}}: en esta tabla (~\ref{datadic:candidate_instances}) se almacenan aquellas instancias que han sido reportadas por un usuario estándar y necesitan ser revisadas por un administrador antes de ser incluidas en el \textit{dataset} utilizable de la aplicación.
	
		\begin{table}
		\small
		\begin{centering}
			\begin{tabular}{@{}p{6em} p{6em} p{6em} p{6em} p{6em}@{}}
				\toprule
				\textbf{Nombre} & \textbf{Tipo de dato y tamaño} & \textbf{Por defecto}& \textbf{Otros} & \textbf{Descripción}\\
				\midrule
				user\_id & INTEGER 8 & anterior & PK & Identificador \\
				instance\_id &  &  &  &  \\
				date &  &  &  &   \\
				suggestion &  &  &  &   \\
				\bottomrule
			\end{tabular}
		\end{centering}
		\caption[Diccionario de datos: Candidate\_instances]{Diccionario de datos: tabla correspondiente a la clase \texttt{Candidate\_instances}.}
		\label{datadic:candidate_instances}
		\end{table}

	\item \textbf{\texttt{Model\_is\_trained\_with}}: en esta tabla (\ref{datadic:modeltrainedwith}) se almacena qué instancia ha sido utilizada para entrenar qué modelo en el sistema.
	
		\begin{table}
		\small
		\begin{centering}
			\begin{tabular}{@{}p{6em} p{6em} p{6em} p{6em} p{6em}@{}}
				\toprule
				\textbf{Nombre} & \textbf{Tipo de dato y tamaño} & \textbf{Por defecto}& \textbf{Otros} & \textbf{Descripción}\\
				\midrule
				model\_id & INTEGER 8 & anterior & PK & Identificador \\
				instance\_id &  &  &  &  \\
				\bottomrule
			\end{tabular}
		\end{centering}
		\caption[Diccionario de datos: Model\_is\_trained\_with]{Diccionario de datos: tabla correspondiente a la clase \texttt{Model\_is\_trained\_with}.}
		\label{datadic:modeltrainedwith}
	\end{table}
\end{itemize}

\section{Diseño procedimental}

\section{Diseño arquitectónico}


