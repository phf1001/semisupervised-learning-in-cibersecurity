\apendice{Especificación de diseño}

\section{Introducción}

\section{Diseño de datos}


\subsection{Diccionario de datos}

Se adjunta a continuación, para cada tabla, el diccionario de datos correspondiente.

\subsubsection{Tabla de usuarios (\texttt{Users})}
Esta tabla (\ref{datadic:users}) almacena la información relevante de cada usuario registrado en la base de datos. Si se comprueba el diagrama E-R, se puede comprobar que una peculiaridad de esta tabla es que es una ISA. Debido a que, además del discriminante (usuario administrador o estándar), únicamente hay un atributo que diferencia ambas clases, se ha decidido implementar utilizando una sola tabla. Cabe destacar que la contraseña está correctamente cifrada mediante  cifrada mediante \texttt{sha-512} y la sal\footnote{\textit{Salt} es un término que se refiere a bits aleatorios usados como entrada (junto a la contraseña) en una función derivadora de claves (cuya salida es la contraseña cifrada).} se ha cifrado en \texttt{sha-256}.

\begin{table}
	\small
	\begin{centering}
		\begin{tabular}{@{}p{6em} p{7em} p{20em}@{}}
			\toprule
			\textbf{Nombre} & \textbf{Tipo} & \textbf{Descripción}\\
			\midrule
			\texttt{\textbf{\underline{id}}} & \texttt{INTEGER \textbf{\underline{PK}}} & Identificador. Generado automáticamente mediante una secuencia. \\
			\texttt{username} &  \texttt{VARCHAR(32) UNIQUE NOT NULL} & Inmutable. Nombre de usuario proporcionado en el registro.\\
			\texttt{email} & \texttt{VARCHAR(256) UNIQUE NOT NULL} & Inmutable. Email del usuario proporcionado en el registro.\\
			\texttt{password} &  \texttt{BYTEA NOT NULL} & Contraseña cifrada mediante \texttt{sha-512}. La \textit{salt} se obtiene aleatoriamente y está cifrada en \texttt{sha-256}.\\
			\texttt{user\_first \_name} & \texttt{VARCHAR(64)} & Nombre del usuario.\\
			\texttt{user\_last \_name} & \texttt{VARCHAR(64)} & Apellidos del usuario. \\
			\texttt{user\_rol} & \texttt{VARCHAR(8)} & Rol del usuario. Puede ser <<standard>> o <<admin>>. Por defecto: <<standard>>. \\
			\texttt{n\_urls \_accepted} & \texttt{INTEGER} &    Atributo de usuarios <<standard>>. Contador de aquellas instancias que han sido reportadas y aceptadas por un administrador. \\
			\bottomrule
		\end{tabular}
	\end{centering}
	\caption[Diccionario de datos: Users]{Diccionario de datos. Tabla correspondiente a la clase \texttt{Users}.}
	\label{datadic:users}
\end{table}

\subsubsection{Tabla de instancias (\texttt{Available\_instances})}

Esta tabla (\ref{datadic:instances}) almacena la información asociada a cada instancia disponible en el \textit{dataset}. Las instancias son URLs que pueden ser utilizadas posteriormente para entrenar y probar modelos. Además, esta tabla también almacena enlaces reportados por usuarios por pertenecer a una lista blanca o negra.

\begin{table}
	\small
	\begin{centering}
		\begin{tabular}{@{}p{7em} p{6em} p{20em}@{}}
			\toprule
			\textbf{Nombre} & \textbf{Tipo} & \textbf{Descripción}\\
			\midrule
			\texttt{\textbf{\underline{instance\_id}}} & \texttt{INTEGER \textbf{\underline{PK}}} & Identificador. Generado automáticamente mediante una secuencia. \\
			\texttt{reviewed\_by} & \texttt{INTEGER FK(Users)} & En caso de estar vacío, significa que la URL todavía no ha sido aprobada por un administrador y está en proceso de revisión, por lo que no se puede utilizar para entrenar modelos.\\
			\texttt{instance\_url} & \texttt{VARCHAR(256) UNIQUE NOT NULL} &  URL de la instancia. Inmutable.\\
			\texttt{instance\_fv} &  \texttt{INTEGER ARRAY[19] NOT NULL} & Vector de características de la URL.   \\
			\texttt{instance \_class} & \texttt{INTEGER NOT NULL} & Etiqueta de la clase. Puede valer $0$ si la URL es legítima o $1$ en caso de que sea \textit{phishing}. \\
			\texttt{colour\_list} & \texttt{VARCHAR(12)} & Puede ser <<black-list>>, <<white-list>> o estar vacío. En caso de contener un valor ha de ser aprobada por un administrador. \\
			\texttt{instance \_labels} & \texttt{CHAR(128)} &  Se preve que el campo varíe bastante, por lo que el tipo de dato no se ha definido como \texttt{VARCHAR}. Contiene etiquetas relacionadas con la instancia.\\
			\bottomrule
		\end{tabular}
	\end{centering}
	\caption[Diccionario de datos: Available\_instances]{Diccionario de datos. Tabla correspondiente a la clase \texttt{Available\_instances}.}
	\label{datadic:instances}
\end{table}


\subsubsection{Tabla de modelos (\texttt{Available\_models})}

Esta tabla (\ref{datadic:models}) almacena los modelos de aprendizaje que han sido creados por un administrador y están disponibles en la \textit{web} para su uso.

\begin{table}
	\small
	\begin{centering}
		\begin{tabular}{@{}p{7em} p{6em} p{20em}@{}}
			\toprule
			\textbf{Nombre} & \textbf{Tipo} & \textbf{Descripción}\\
			\midrule
			\texttt{\textbf{\underline{model\_id}}} & \texttt{INTEGER \textbf{\underline{PK}}} & Identificador. Generado automáticamente mediante una secuencia. \\
			\texttt{created\_by} & \texttt{INTEGER FK(Users) NOT NULL} &  Enlace al administrador que ha creado un determinado modelo. Dependencia por existencia.\\
			\texttt{model\_name} & \texttt{VARCHAR(32) UNIQUE NOT NULL} & Inmutable. Nombre común y visible para los usuarios. \\
			\texttt{file\_name} & \texttt{VARCHAR(32) UNIQUE NOT NULL} & Inmutable. Preferiblemente seguirá la sintáxis \texttt{nombre\_v-a-b-c.pkl}, donde $a$, $b$, $c$ son los números de versión (ejemplo: \texttt{cof\_v-1-0-0.pkl})    \\
			\texttt{creation\_date} & \texttt{TIMESTAMP} &  Fecha de creación.   \\
			\texttt{is\_default} & \texttt{BOOLEAN NOT NULL} & Por defecto <<false>>. Determina si un clasificador es el utilizado por defecto (cuando un usuario realiza una consulta y no selecciona ningún modelo).   \\
			\texttt{is\_visible} & \texttt{BOOLEAN NOT NULL} & Por defecto <<true>>. Determina si un clasificador está visible en la \textit{web} y disponible para ser utilizado.\\
			\texttt{model\_scores} & \texttt{FLOAT(3) ARRAY NOT NULL} & Por defecto \texttt{[0.0, 0.0, 0.0]}. Son las últimas estadísticas del último \textit{test} realizado al clasificador. Las columnas (¡en orden!) corresponden a las métricas \textit{accuracy}, \textit{precision} y \textit{recall}. \\			
			\texttt{model\_notes} & \texttt{VARCHAR(128)} & Notas acerca de los clasificadores base utilizados para crear el \textit{ensemble}. No debería cambiar de valor.\\
			\texttt{random\_state} & \texttt{INTEGER} &  Semilla aleatoria.  \\
			\bottomrule
		\end{tabular}
	\end{centering}
	\caption[Diccionario de datos: Available\_models]{Diccionario de datos: tabla correspondiente a la clase \texttt{Available\_models}.}
	\label{datadic:models}
\end{table}

Si se comprueba el diagrama E-R, se puede comprobar que esta tabla es una ISA. En este caso se ha decidido implementar considerando que es una interrelación 1:1. Por ello, se ha creado una tabla para la generalización y otra tabla para cada una de las especializaciones. Estas se enumeran a continuación:

\begin{itemize}
	\item \textbf{\texttt{Available\_co\_forests}}: los modelos pertenecientes a este algoritmo poseen los atributos de la tabla~\ref{datadic:coforest}.
	
	\begin{table}
		\small
		\begin{centering}
			\begin{tabular}{@{}p{6em} p{9em} p{19em}@{}}
				\toprule
				\textbf{Nombre} & \textbf{Tipo} & \textbf{Descripción}\\
				\midrule
				\texttt{\textbf{\underline{model\_id}}} & \texttt{INTEGER \textbf{\underline{PK}}, FK(Available \_models)} & Identificador. Generado por la secuencia de la tabla padre e introducido mediante código (debe coincidir). \\
				\texttt{max\_features} & \texttt{VARCHAR(4) NOT NULL} & Parámetro del árbol de decisión. Puede ser <<sqrt>> o <<log2>>. Por defecto <<log2>>.   \\
				\texttt{thetha} & \texttt{FLOAT(3) NOT NULL} & Parámetro thetha del \textit{ensemble} (umbral de confianza). Por defecto 0.75.   \\
				\texttt{n\_trees} & \texttt{INTEGER NOT NULL} & Número de árboles. Por defecto 3.   \\
				\bottomrule
			\end{tabular}
		\end{centering}
		\caption[Diccionario de datos: Available\_co\_forests]{Diccionario de datos: tabla correspondiente a la clase \texttt{Available\_co\_forests}.}
		\label{datadic:coforest}
	\end{table}

	\item \textbf{\texttt{Available\_tri\_trainings}}: los modelos pertenecientes a este algoritmo poseen los atributos de la tabla~\ref{datadic:tritraining}.

	\begin{table}
		\small
		\begin{centering}
			\begin{tabular}{@{}p{6em} p{12em} p{16em}@{}}
				\toprule
				\textbf{Nombre} & \textbf{Tipo} & \textbf{Descripción}\\
				\midrule
				\texttt{\textbf{\underline{model\_id}}} & \texttt{INTEGER \textbf{\underline{PK}}, FK(Available \_models)} & Identificador. Generado por la secuencia de la tabla padre e introducido mediante código (debe coincidir).\\
				\texttt{cls\_one} & \texttt{VARCHAR(4) NOT NULL}  &  Algoritmo del primer estimador base. Puede ser <<kNN>>, <<NB>> o <<tree>>.\\
				\texttt{cls\_two} & \texttt{VARCHAR(4) NOT NULL} & Algoritmo del segundo estimador base. Puede ser <<kNN>>, <<NB>> o <<tree>>.\\
				\texttt{cls\_three} & \texttt{VARCHAR(4) NOT NULL} & Algoritmo del tercer estimador base. Puede ser <<kNN>>, <<NB>> o <<tree>>.\\
				\bottomrule
			\end{tabular}
		\end{centering}
		\caption[Diccionario de datos: Available\_tri\_trainings]{Diccionario de datos: tabla correspondiente a la clase \texttt{Available\_tri\_trainings}.}
		\label{datadic:tritraining}
	\end{table}

	\item \textbf{\texttt{Available\_democratic\_cos}}: los modelos pertenecientes a este algoritmo poseen los atributos de la tabla~\ref{datadic:democraticco}.

	\begin{table}
		\small
		\begin{centering}
			\begin{tabular}{@{}p{6em} p{8em} p{19em}@{}}
				\toprule
				\textbf{Nombre} & \textbf{Tipo} & \textbf{Descripción}\\
				\midrule
				\texttt{\textbf{\underline{model\_id}}} & \texttt{INTEGER \textbf{\underline{PK}}, FK(Available \_models)} & Identificador. Generado por la secuencia de la tabla padre e introducido mediante código (debe coincidir).\\
				\texttt{n\_clss} & \texttt{INTEGER NOT NULL} & Número de clasificadores base.\\
				\texttt{base\_clss} & \texttt{VARCHAR(4) ARRAY NOT NULL}  & Array con los algoritmos pertenecientes a los estimadores base. Los valores que puede contener son <<kNN>>, <<NB>> o <<tree>>.\\
				\bottomrule
			\end{tabular}
		\end{centering}
		\caption[Diccionario de datos: Available\_democratic\_cos]{Diccionario de datos: tabla correspondiente a la clase \texttt{Available\_democratic\_cos}.}
		\label{datadic:democraticco}
	\end{table}
\end{itemize}

\subsubsection{Tablas de relación}

Se enumeran a continuación las tablas que son resultado de una relación entre entidades.

\begin{itemize}
	\item \textbf{\texttt{Candidate instances}}: en esta tabla (~\ref{datadic:candidate_instances}) se almacenan aquellas instancias que han sido reportadas por un usuario estándar y necesitan ser revisadas por un administrador antes de ser incluidas en el \textit{dataset} utilizable de la aplicación.
	
		\begin{table}
		\small
		\begin{centering}
			\begin{tabular}{@{}p{7em} p{7em} p{18em}@{}}
				\toprule
				\textbf{Nombre} & \textbf{Tipo} & \textbf{Descripción}\\
				\midrule
				\texttt{user\_id} & \texttt{INTEGER FK(Users)} & Identificador del usuario que ha reportado la URL. \\
				\texttt{instance\_id} & \texttt{INTEGER FK(Available \_instances)} & Identificador de la URL reportada. \\
				\texttt{date\_reported} & \texttt{TIMESTAMP} & Fecha de envío del formulario.     \\
				\texttt{suggestion} & \texttt{VARCHAR(16) NOT NULL} & Lo que el usuario denuncia que la URL es (ejemplo: <<white-list>>).   \\\\
				\multicolumn{3}{l}{\texttt{\textbf{\underline{PK(user\_id, instance\_id)}}} Clave primaria compuesta de la relación.} \\
				\bottomrule
			\end{tabular}
		\end{centering}
		\caption[Diccionario de datos: Candidate\_instances]{Diccionario de datos: tabla correspondiente a la clase \texttt{Candidate\_instances}.}
		\label{datadic:candidate_instances}
		\end{table}

	\item \textbf{\texttt{Model\_is\_trained\_with}}: en esta tabla (\ref{datadic:modeltrainedwith}) se almacena qué instancia ha sido utilizada para entrenar qué modelo en el sistema.
	
		\begin{table}
		\small
		\begin{centering}
			\begin{tabular}{@{}p{7em} p{7em} p{18em}@{}}
				\toprule
				\textbf{Nombre} & \textbf{Tipo} & \textbf{Descripción}\\
				\midrule
				\texttt{model\_id} & \texttt{INTEGER FK(Available \_models)} & Identificador del modelo que ha sido entrenado. \\
				\texttt{instance\_id} & \texttt{INTEGER FK(Available \_instances)} & Identificador de la instancia utilizada en la fase de entrenamiento. \\\\\\
				\multicolumn{3}{l}{\texttt{\textbf{\underline{PK(model\_id, instance\_id)}}} Clave primaria compuesta de la relación.} \\
				\bottomrule
			\end{tabular}
		\end{centering}
		\caption[Diccionario de datos: Model\_is\_trained\_with]{Diccionario de datos: tabla correspondiente a la clase \texttt{Model\_is\_trained\_with}.}
		\label{datadic:modeltrainedwith}
	\end{table}
\end{itemize}

\section{Diseño procedimental}

\section{Diseño arquitectónico}


