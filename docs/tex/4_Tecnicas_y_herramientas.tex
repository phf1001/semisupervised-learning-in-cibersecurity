\capitulo{4}{Técnicas y herramientas}

\section{Herramientas}

\subsection{Librerías}

\subsection{Scripts} 
\subsubsection{Script para levantar proxies \textit{SOCKS5}}

Durante la extracción de vectores de características, se realizan peticiones a páginas de \textit{phishing}. Para garantizar que estas páginas no puedan rastrear desde donde se ha realizado la petición, se han utilizado \textit{proxies} que implementan el protocolo SOCKS5.

Para ello, se ha implementado en python un \textit{script} auxiliar que levanta en paralelo tantas instancias de \textit{Tor} como se soliciten, y mantiene los hilos vivos hasta que se interrumpa la ejecución del \textit{script}.

Por cada instancia de Tor que se quiera levantar, se necesita un fichero \texttt{torrc} en el directorio \texttt{/etc/tor/}~\cite{TorFicherosTor}. Cada una de ellas debe tener, además, su propio puerto de control, su propio puerto \textit{socks} y su directorio de datos. Por ello, se ha creado una clase auxiliar que genera estos ficheros. Para levantar la instancia simplemente ha de ejecutarse el comando \texttt{tor -f /etc/tor/torrc.x} (siendo $x$ el número de la instacia correspondiente), aunque se ha decidido, además, dirigir la salida al fichero \texttt{/dev/null}. Es importante destacar que el puerto de control debe ser el siguiente al puerto \textit{socks}. Teniendo en cuenta que los puertos por defecto de Tor son el 9050 y el 9051, se puede incrementar partiendo de esos números~\cite{TorficheroComando}. Para comprobar que la instancia levantada funciona correctamente, se hace una petición a <<http://ipinfo.io/ip>> y se comprueba con una expresión regular que la dirección obtenida es la correcta. De este modo, se sabe que el \textit{proxy http} levantado funciona, y se puede utilizar redireccionando las peticiones oportunas a través del \textit{proxy} \texttt{socks5h://127.0.0.1:$y$} (donde $y$ es el puerto \textit{socks} de la instancia correspondiente). 

Debido a que el puerto Tor se abre <<a la escucha>> en la máquina local (está esperando que se realice una conexión), no conlleva ningún riesgo. Para cerrar el puerto, es suficiente con parar el proceso que esté ejecutando el \textit{script}. Se puede comprobar ejecutando en una terminal el comando \texttt{sudo lsof -i:$y$} (nuevamente, $y$ es el puerto \textit{socks} levantado). Cuando el \textit{script} esté funcionando, la salida del comando mostrará diversos campos, como el propio comando (Tor), el \textit{PID}, el nombre (\textit{listen}). Si el \textit{script} se para, el comando no mostrará salida, lo que implica que el puerto no está abierto~\cite{checkOpenTorPorts}.