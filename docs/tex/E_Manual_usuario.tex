\apendice{Documentación de usuario}

\section{Introducción}

En este anexo se pretende describir de forma concisa las características y funcionalidades de la \textit{web} desarrollada, además de los requisitos necesarios para su correcta renderización.

Es importante destacar que la \textit{web} ha sido desplegada en Heroku. Sin embargo, esta PaaS, tiene importantes restricciones en las versiones para estudiantes. En concreto, Krini se ve gravemente perjudicada ya que Heroku cancela toda petición que dure más de 30 segundos (lo que impide, en la mayoría de los enlaces, realizar un análisis completo).

Por este motivo, también se va a documentar cómo desplegar la aplicación en local (servidor de producción y base de datos) mediante contenedores de Docker.

\section{Requisitos de usuarios}

En este apartado se pretenden enumerar los requisitos necesarios para acceder correctamente a la \textit{web} desarrollada.

\subsection{Requisitos Heroku}
\label{s-e:requisitos-heroku}

Los requisitos en este caso vienen limitados por el uso de las bibliotecas \texttt{Chart.js} (v2.9.3) y \texttt{Bootstrap} (v4.4.1). En este caso, se necesita que la \textit{web} se renderice en navegadores con las siguientes versiones:

\begin{itemize}
	\item \textbf{Google Chrome:} todas las versiones modernas.
	\item \textbf{Mozilla Firefox:} todas las versiones modernas.
	\item \textbf{Safari:} todas las versiones modernas.
	\item \textbf{Microsoft Edge:} todas las versiones modernas.
	\item \textbf{Internet Explorer:} versión 10 o superior.
\end{itemize}

\subsection{Requisitos Docker}
\label{s-e:requisitos-docker}

Además de los requisitos expuestos en la sección~\ref{s-e:requisitos-heroku}, si se quiere ejecutar el contenedor de Docker en local, se ha de cumplir con las siguientes características en función del sistema operativo. Es relevante que, como se puede comprobar en el diagrama de despliegue de Docker (disponible en la ilustración~\ref{c:diagrama-deploy-docker}), los contenedores de la base de datos y de la \textit{web} son independientes. Por ello, se ha de contar con el \textit{plugin} \texttt{docker-compose}.

\begin{enumerate}
	\item \textbf{Windows}: se recomienda utilizar Docker Desktop por su simplicidad. Los requisitos completos se pueden consultar en su documentación oficial\footnote{Disponible en \url{https://docs.docker.com/desktop/install/windows-install/}}. Sin embargo, se resumen a continuación:
	
	\begin{itemize}
		\item \texttt{WLS}
		\item Procesador de 64 bits
		\item 4GB de memoria RAM
		\item 3GB de memoria para las imágenes
		\item El \textit{plugin} \texttt{docker-compose} viene por defecto con la versión de escritorio.
	\end{itemize}

	\item \textbf{Linux}: nuevamente, los requisitos completos se facilitan en la documentación\footnote{Disponible en \url{https://docs.docker.com/desktop/install/linux-install/}}. En resumen, se recomienda:
	\begin{itemize}
		\item Disponer de soporte para virtualización
		\item Procesador de 64 bits
		\item 4GB de memoria RAM
		\item 3GB de memoria para las imágenes
		\item Instalar el \textit{plugin} \texttt{docker-compose} mediante \texttt{\$ sudo apt-get install docker-compose-plugin} (en Ubuntu y Debian) o \texttt{\$ sudo yum install docker-compose-plugin} (en distribuciones basadas en RPM).
	\end{itemize}
\end{enumerate}



\section{Instalación}

La instalación de un producto \textit{software} es el proceso mediante el cual se configura y prepara un programa o aplicación para que pueda ser utilizado en un dispositivo objetivo.

Debido a que la aplicación desarrollada es una \textit{web}, no hace falta pasar por este proceso. Sin embargo y debido a las restricciones de Heroku anteriormente mencionadas, se va a explicar cómo desplegar la aplicación en local mediante Docker (con servidores de producción) para comprobar la funcionalidad al completo. Se recuerda que las instrucciones para levantar el servidor de desarrollo se encuentran en la sección~\ref{s-d:flask-deploy}.

\subsection{Acceso mediante Heroku}

Simplemente se ha de acceder mediante el navegador introduciendo la dirección \url{https://krini.herokuapp.com/}

\subsection{Despliegue en Docker}
\label{s-e:docker-deploy-users}

Para facilitar que cualquier usuario pueda desplegar un contenedor de Docker (en realidad, dos) y levantar su propio servidor de gunicorn en local sin tener conocimientos técnicos, se han preparado unos \textit{scripts} multiplataforma disponibles en \url{https://github.com/phf1001/semisupervised-learning-in-cibersecurity/tree/main/docker-deploy-kit}

De esta forma, tan solo se debe seleccionar el sistema operativo anfitrión, descargar los archivos (se facilita un comprimido con todos incluidos) y garantizar que se cumple con los requisitos de la sección~\ref{s-e:requisitos-docker}.

Para levantar el servidor en local y garantizar que la base de datos se rellena completamente, se han de seguir los siguientes pasos teniendo en cuenta que los \textit{scripts} en Linux se ejecutan mediante el comando \texttt{\$ sh nombre-script.sh} y en Windows mediante \texttt{\$ nombre-script.bat}.

\begin{enumerate}
	\item Ubicarse en la carpeta donde se encuentren los \textit{scripts}.
	\item Si es la primera vez que se ejecuta:
	\begin{enumerate}
	\item Lanzar el script \texttt{docker-first-time-1.sh} y seguir las instrucciones (esperar 30 segundos y abrir el navegador cuando lo indique).
	\item Ejecutar el \textit{script} \texttt{docker-first-time-2.sh} y respetar los pasos indicados (esperar 30 segundos antes de abrir el navegador).
	\end{enumerate}
	\item Si se quieren parar los contenedores para reutilizarlos luego, ejecutar el \textit{script} \texttt{docker-stop.sh} y \texttt{docker-start.sh} para volver a iniciarlos.
	\item Si se quieren borrar las imágenes y contenedores del sistema definitivamente, ejecutar el \textit{script} \texttt{docker-clean.sh}
\end{enumerate}

\textbf{Importante}: iniciar el demonio de Docker antes de lanzar los \textit{scripts}. En Windows basta con iniciar la aplicación de escritorio.

\section{Manual del usuario}

En este punto del manual se supone que todos los usuarios tienen acceso a la aplicación, ya sea mediante un contenedor de Docker o Heroku.

A continuación, se procede a ilustrar las funcionalidades básicas de la \textit{web}. Es destacable que no todos los usuarios cuentan con acceso a todas las funcionalidades (consultar diagrama de casos de uso~\ref{b:diagrama-cu}). Por ello, se recomienda probar el producto con una cuenta con permisos de administración.


\subsection{Analizador}

\begin{figure}[h]
	\caption[Manual de usuario: página principal]{Página principal de la aplicación.}
	\centering
	\includegraphics[width=\textwidth]{../img/anexos/user_guide/1_index}
	\label{e-1:index}
\end{figure}

En la ilustración~\ref{e-1:index} se muestra la página principal del analizador disponible para cualquier usuario. Para escanear una URL, basta con introducirla en el campo correspondiente. Es preferible introducir la dirección tal y como aparece en el navegador, pero si no es posible, la aplicación tratará de completarla.

A continuación se pueden seleccionar los clasificadores que se desee en el desplegable. Si no se selecciona ninguno, se analizará con el modelo por defecto, y en caso de que no exista, con uno aleatorio.

Opcionalmente se puede marcar la casilla <<análisis rápido>>. En este caso, no se extraerá el vector de características si ya existe en la base de datos.

Tras pulsar el botón de analizar, se mostrará una pantalla de carga mientras se extrae el vector de características correspondiente.

\begin{figure}[h]
	\caption[Manual de usuario: página de carga]{Página de análisis de URL.}
	\centering
	\includegraphics[width=\textwidth]{../img/anexos/user_guide/2_analysis}
	\label{e-1:analysis}
\end{figure}

\begin{figure}[h]
	\caption[Manual de usuario: error 403]{Página de error 403.}
	\centering
	\includegraphics[width=\textwidth]{../img/anexos/user_guide/0_error_403}
	\label{e-0:error-403}
\end{figure}

\begin{figure}[h]
	\caption[Manual de usuario: error 404]{Página de error 404.}
	\centering
	\includegraphics[width=\textwidth]{../img/anexos/user_guide/0_error_404}
	\label{e-0:error-404}
\end{figure}

\begin{figure}[h]
	\caption[Manual de usuario: error 500]{Página de error 500.}
	\centering
	\includegraphics[width=\textwidth]{../img/anexos/user_guide/0_error_500}
	\label{e-0:error-500}
\end{figure}

\begin{figure}[h]
	\caption[Manual de usuario: gráficas del \textit{dashboard}]{\textit{Dashboard} mostrado tras un análisis.}
	\centering
	\includegraphics[width=\textwidth]{../img/anexos/user_guide/3_dashboard_1}
	\label{e-3:dashboard-1}
\end{figure}

\begin{figure}[h]
	\caption[Manual de usuario: vector de características en el \textit{dashboard}]{Análisis del vector de características mostrado en el dashboard.}
	\centering
	\includegraphics[width=\textwidth]{../img/anexos/user_guide/3_dashboard_2}
	\label{e-3:dashboard-2}
\end{figure}

\begin{figure}[h]
	\caption[Manual de usuario: reportar análisis erróneo]{Reportar análisis realizados erróneamente.}
	\centering
	\includegraphics[width=\textwidth]{../img/anexos/user_guide/3_report_false_analysis}
	\label{e-3:report-false-analysis}
\end{figure}

\begin{figure}[h]
	\caption[Manual de usuario: reportar pertenencia a lista]{Página para reportar pertenencia a lista blanca o lista negra.}
	\centering
	\includegraphics[width=\textwidth]{../img/anexos/user_guide/4_report_url}
	\label{e-3:report-url}
\end{figure}

\begin{figure}[h]
	\caption[Manual de usuario: página de modelos]{Página de administración de modelos.}
	\centering
	\includegraphics[width=\textwidth]{../img/anexos/user_guide/5_models}
	\label{e-5:models}
\end{figure}

\begin{figure}[h]
	\caption[Manual de usuario: nuevo modelo]{Formulario de creación de nuevos modelos.}
	\centering
	\includegraphics[width=\textwidth]{../img/anexos/user_guide/5_new_model}
	\label{e-5:new-model}
\end{figure}

\begin{figure}[h]
	\caption[Manual de usuario: editar modelo]{Formulario de edición de modelos existentes.}
	\centering
	\includegraphics[width=\textwidth]{../img/anexos/user_guide/5_edit_model}
	\label{e-5:edit-model}
\end{figure}

\begin{figure}[h]
	\caption[Manual de usuario: evaluar modelo]{Página de evaluación de modelos existentes.}
	\centering
	\includegraphics[width=\textwidth]{../img/anexos/user_guide/5_test_model}
	\label{e-5:test-model}
\end{figure}

\begin{figure}[h]
	\caption[Manual de usuario: página de instancias]{Página de administración de instancias.}
	\centering
	\includegraphics[width=\textwidth]{../img/anexos/user_guide/6_instances}
	\label{e-5:instances}
\end{figure}

\begin{figure}[h]
	\caption[Manual de usuario: nueva instancia]{Formulario de creación de nuevas instancias.}
	\centering
	\includegraphics[width=\textwidth]{../img/anexos/user_guide/6_new_instance}
	\label{e-5:new-instance}
\end{figure}

\begin{figure}[h]
	\caption[Manual de usuario: editar instancia]{Formulario de edición de instancias existentes.}
	\centering
	\includegraphics[width=\textwidth]{../img/anexos/user_guide/6_edit_instance}
	\label{e-5:edit-instance}
\end{figure}

\begin{figure}[h]
	\caption[Manual de usuario: etiquetas predeterminadas]{Menú con las etiquetas predeterminadas.}
	\centering
	\includegraphics[width=\textwidth]{../img/anexos/user_guide/6_labels}
	\label{e-6:labels}
\end{figure}

\begin{figure}[h]
	\caption[Manual de usuario: ejemplos de etiquetas]{Instancias con variedad de etiquetas.}
	\centering
	\includegraphics[width=\textwidth]{../img/anexos/user_guide/6_instances_more_labels}
	\label{e-6:more-labels}
\end{figure}

\begin{figure}[h]
	\caption[Manual de usuario: administrar sugerencias]{Página de administración de sugerencias o \textit{reports}.}
	\centering
	\includegraphics[width=\textwidth]{../img/anexos/user_guide/7_reports}
	\label{e-7:reports}
\end{figure}

\begin{figure}[h]
	\caption[Manual de usuario: inicio de sesión]{Página de inicio de sesión.}
	\centering
	\includegraphics[width=\textwidth]{../img/anexos/user_guide/8_login}
	\label{e-8:login}
\end{figure}

\begin{figure}[h]
	\caption[Manual de usuario: crear nueva cuenta]{Página de creación de nuevas cuentas.}
	\centering
	\includegraphics[width=\textwidth]{../img/anexos/user_guide/8_register}
	\label{e-8:register}
\end{figure}

\begin{figure}[h]
	\caption[Manual de usuario: perfil]{Perfil de un usuario que ha iniciado sesión en la aplicación.}
	\centering
	\includegraphics[width=\textwidth]{../img/anexos/user_guide/8_profile}
	\label{e-8:profile}
\end{figure}

\begin{figure}[h]
	\caption[Manual de usuario: barra navegación (usuario iniciado)]{Barra de navegación correspondiente a un administrador}
	\centering
	\includegraphics[width=\textwidth]{../img/anexos/user_guide/9_navbar_init}
	\label{e-9:navbar}
\end{figure}

\begin{figure}[h]
	\caption[Manual de usuario: barra navegación (visitante)]{Barra de navegación correspondiente a un usuario visitante.}
	\centering
	\includegraphics[width=\textwidth]{../img/anexos/user_guide/9_navbar_no_init}
	\label{e-9:navbar-2}
\end{figure}

\begin{figure}[h]
	\caption[Manual de usuario: \textit{dashboard} (versión móvil)]{\textit{Dashboard} visualizado desde el navegador de un teléfono.}
	\centering
	\includegraphics[scale=0.1]{../img/anexos/user_guide/0_dashboard_mobile}
	\label{e-0:dashboard-mobile}
\end{figure}

\begin{figure}[h]
	\caption[Manual de usuario: menú (versión móvil)]{Menú de navegación visualizado desde un dispositivo móvil.}
	\centering
	\includegraphics[scale=0.1]{../img/anexos/user_guide/0_menu_mobile}
	\label{e-0:menu-mobile}
\end{figure}