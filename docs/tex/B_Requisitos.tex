\apendice{Especificación de Requisitos}

\section{Introducción}

\section{Objetivos generales}

\section{Catálogo de requisitos}

\section{Especificación de requisitos}

COMENTAR: Si estamos haciendo Scrum deberían ser historias de usuario (?)

Se detalla a continuación la especificación de requisitos de la aplicación \textit{web} propuesta.

\subsection{Diagrama de casos de uso}

Partiendo de las historias de usuario extraídas en las entrevistas realizadas con el \textit{product owner}, se ha extraído el siguiente diagrama de casos de uso.

[Explicación general + diagrama]

A continuación se muestra la tabla correspondiente a cada caso de uso.

% Caso de Uso 1 -> Escanear URL
\begin{table}[p]
	\centering
	\begin{tabularx}{\linewidth}{ p{0.21\columnwidth} p{0.71\columnwidth} }
		\toprule
		\textbf{CU-1}    & \textbf{Escanear URL}\\
		\toprule
		\textbf{Versión}              & 1.0    \\
		\textbf{Autor}                & Patricia Hernando Fernández \\
		\textbf{Requisitos asociados} & RF-xx, RF-xx \\
		\textbf{Descripción}          & Se permite que un usuario realice un escaneo de la URL que desee. Para ello, seleccionará los modelos que considere adecuados (consultar CU-1.1 en la tabla~\ref{cu:seleccionar-modelos-ml}) y se le mostrarán los resultados en un \textit{dashboard} (CU-2 en la tabla~\ref{cu:visualizar-resultados}). \\
		\textbf{Precondición}         & No hay ninguna precondición. \\
		\textbf{Acciones}             &
		\begin{enumerate}
			\def\labelenumi{\arabic{enumi}.}
			\tightlist
			\item El usuario introduce la URL.
			\item El usuario selecciona los modelos de ML que considere (consultar CU-1.1 en la tabla~\ref{cu:seleccionar-modelos-ml}).
			\item El usuario decide si quiere realizar un análisis rápido.
			\item El usuario pulsa el botón de <<realizar análisis>>.
		\end{enumerate}\\
		\textbf{Postcondición}        & ...TODO \\
		\textbf{Excepciones}          & ...TODO \\
		\textbf{Importancia}          & ...TODO \\
		\bottomrule
	\end{tabularx}
	\caption{CU-1 Escanear URL.}
	\label{cu:escanear-url}
\end{table}


% Caso de Uso 1.1 -> Seleccionar modelos
\begin{table}[p]
	\centering
	\begin{tabularx}{\linewidth}{ p{0.21\columnwidth} p{0.71\columnwidth} }
		\toprule
		\textbf{CU-1.1}    & \textbf{Seleccionar modelos con los que analizar}\\
		\toprule
		\textbf{Versión}              & 1.0    \\
		\textbf{Autor}                & Patricia Hernando Fernández \\
		\textbf{Requisitos asociados} & RF-xx, RF-xx \\
		\textbf{Descripción}          & Se mostrará al usuario un desplegable con los distintos modelos disponibles y visibles en la base de datos. \\
		\textbf{Precondición}         & Estar realizando el CU-1 (disponible en la tabla~\ref{cu:escanear-url}). \\
		\textbf{Acciones}             &
		\begin{enumerate}
			\def\labelenumi{\arabic{enumi}.}
			\tightlist
			\item El usuario despliega la lista de modelos disponibles.
			\item El usuario selecciona los modelos que quiere utilizar.
			\item El usuario cierra el despegable.
		\end{enumerate}\\
		\textbf{Postcondición}        & Los modelos seleccionados quedan guardados. \\
		\textbf{Excepciones}          & ...TODO Ninguna (en caso de no haber clasificadores, se controla la excepción en el índice). \\
		\textbf{Importancia}          & Baja \\
		\bottomrule
	\end{tabularx}
	\caption{CU-1.1 Seleccionar modelos con los que analizar.}
	\label{cu:seleccionar-modelos-ml}
\end{table}

% Caso de Uso 2 -> Visualizar resultados
\begin{table}[p]
	\centering
	\begin{tabularx}{\linewidth}{ p{0.21\columnwidth} p{0.71\columnwidth} }
		\toprule
		\textbf{CU-2}    & \textbf{Visualizar resultados}\\
		\toprule
		\textbf{Versión}              & 1.0    \\
		\textbf{Autor}                & Patricia Hernando Fernández \\
		\textbf{Requisitos asociados} & RF-xx, RF-xx \\
		\textbf{Descripción}          & Se muestra al usuario los resultados tras haber analizado la URL mediante los algoritmos de ML seleccionados. Para ello se hará uso de un \textit{dashboard}, donde además se podrá notificar un falso resultado (CU-2.1 en la tabla~\ref{cu:notificar-falso-resultado}).\\
		\textbf{Precondición}         & Haber realizado un análisis previamente (CU-1 en la tabla~\ref{cu:escanear-url}). \\
		\textbf{Acciones}             &
		\begin{enumerate}
			\def\labelenumi{\arabic{enumi}.}
			\tightlist
			\item El usuario es redirigido a un \textit{dashboard}.
			\item El usuario puede interactuar con los distintos gráficos mostrados en la página.
			\item El usuario puede notificar si considera que el análisis es erróneo (CU-2.1 en la tabla~\ref{cu:notificar-falso-resultado}).
		\end{enumerate}\\
		\textbf{Postcondición}        & ...TODO Postcondiciones (podría haber más de una) \\
		\textbf{Excepciones}          & ...TODO Excepciones \\
		\textbf{Importancia}          & Alta \\
		\bottomrule
	\end{tabularx}
	\caption{CU-2 Visualizar resultados.}
	\label{cu:visualizar-resultados}
\end{table}


% Caso de Uso 2.1 -> Notificar falso resultado
\begin{table}[p]
	\centering
	\begin{tabularx}{\linewidth}{ p{0.21\columnwidth} p{0.71\columnwidth} }
		\toprule
		\textbf{CU-2.1}    & \textbf{Notificar falso resultado}\\
		\toprule
		\textbf{Versión}              & 1.0    \\
		\textbf{Autor}                & Patricia Hernando Fernández \\
		\textbf{Requisitos asociados} & RF-xx, RF-xx \\
		\textbf{Descripción}          & Permite a un usuario notificar automáticamente a la aplicación si considera que el resultado es erróneo.\\
		\textbf{Precondición}         & Encontrarse en el \textit{dashboard} tras un análisis (CU-2 en la tabla~\ref{cu:visualizar-resultados}). \\
		\textbf{Acciones}             &
		\begin{enumerate}
			\def\labelenumi{\arabic{enumi}.}
			\tightlist
			\item El usuario pulsa el botón <<notificar falso resultado>>.
		\end{enumerate}\\
		\textbf{Postcondición}        & ...TODO Se registra la URL reportada en la base de datos con el campo <<sugerencia>> establecido al valor opuesto del resultado mostrado. \\
		\textbf{Excepciones}          & ...TODO \\
		\textbf{Importancia}          & Media \\
		\bottomrule
	\end{tabularx}
	\caption{CU-2.1  Notificar falso resultado.}
	\label{cu:notificar-falso-resultado}
\end{table}