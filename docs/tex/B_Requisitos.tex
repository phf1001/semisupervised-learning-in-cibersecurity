\apendice{Especificación de Requisitos}

\section{Introducción}

La fase de análisis es una etapa fundamental en el ciclo de vida del desarrollo ya que permite entender los requerimientos del cliente e identificar los componentes necesarios para entregar un producto adecuado.

Durante esta fase, el equipo de desarrollo se reune con el cliente (o con su representante el \textit{product owner}) y, tras realizar diversas entrevistas, se recoge lo que este espera de la aplicación. Además, el equipo de desarrollo analizará todos los requisitos no relacionados con la funcionalidad (como puede ser seguridad, escalabilidad, rendimiento, etc.) que se puedan deducir de dichas conversaciones.

Debido a la importancia que tiene esta fase, se ha experimentado una evolución con los años y se han propuesto distintas aproximaciones en función de las metodologías utilizadas. Entre ellas:

\begin{itemize}
	\item \textbf{Metodologías tradicionales}: estas metodologías realizan la captura de los requerimientos en fases tempranas del desarrollo, realizando una <<fotografía>> exacta (e, idealmente, inmutable) de lo que el usuario necesita. Para ello, se hace uso de requisitos funcionales y no funcionales que se documentan en un formato estructurado y definido en estándares. Un ejemplo de especificación es el estándar IEEE 830~\cite{ieee830}, donde se implora que los requisitos han de ser claros, precisos, medibles, coherentes, completos, factibles, específicos y verificables.
	
	\item  \textbf{Metodologías ágiles}: en contraposición, las metodologías ágiles tratan de ser flexibles y adaptarse al usuario (ya que, muchas veces, es común que no tenga claro en etapas tempranas de desarrollo lo que realmente necesita). Para ello utilizan las denominadas historias de usuario, que son breves descripciones de las funcionalidades que el usuario necesita para realizar una tarea específica. Estos documentos se recogen en estructuras tabulares y se almacenan en el \textit{product backlog}, que lejos de ser una <<fotografía>>, es una lista <<viva>> que se actualiza y prioriza en función de las necesidades del cliente. 
\end{itemize}


Aunque en un proyecto real puede no resultar óptimo realizar una <<mezcla>> de metodologías, debido a las peculiares características de este (equipo de un desarrollador) y a que se solicita la inclusión de requisitos funcionales y no funcionales de una manera más tradicional en la memoria, se ha hecho uso de ambos métodos de documentación (complementados, además, mediante prototipos expuestos en la sección~\ref{s:mockups}). Por lo tanto, se realizará una especificación de requisitos clásica (expuesta en la sección~\ref{s:cat-requisitos} y desarrollada en la sección~\ref{s:requisitos}) complementada con alguna historia de usuario extraída de las entrevistas con el \textit{product owner} (sección~\ref{s:hu}).

Cabe destacar que durante el desarrollo real se ha utilizado una metodología ágil mediante el uso de \textit{sprints} como se ha expuesto en la sección~\ref{s:planificacion-sprints}.

\section{Objetivos generales}

En este proyecto se han definido distintos objetivos que pueden ser resumidos en los siguientes puntos:

\begin{enumerate}
	\item Implementación y validación de algoritmos de aprendizaje semisupervisado: en concreto el \textit{co-forest}, el \textit{democratic-co learning} y el \textit{tri-training}.
	\item Aplicación del \textit{machine learning} a la solución de un problema real relacionado con la ciberseguridad: se ha escogido la detección de \textit{phishing} y detección de ataques en sistemas de recomendación, aunque esta última ha sido descartada por su bajo desempeño.
	\item Desarrollo de una herramienta que permita poner al servicio de la comunidad el conocimiento desarrollado: se ha decidido desarrollar un analizador de \textit{phishing} en formato página \textit{web}, permitiendo además administración avanzada de modelos de \textit{machine learning} e instancias de aprendizaje (URLs legítimos y fraudulentos).
\end{enumerate}

Como se puede intuir, los dos primeros están enfocados a la investigación mientras que el último es una tarea de desarrollo. Por este motivo, este anexo se centrará principalmente en el tercer punto.

\section{Usuarios}
\label{s:usuarios}

En las metodologías ágiles no sólo es necesario definir las funcionalidades, sino también quién las realiza. Por ello, se introducen a continuación los distintos usuarios de la página \textit{web} desarrollada, aunque sus funciones completas se pueden observar en el diagrama de casos de uso~\ref{b:diagrama-cu}.

\begin{itemize}
	\item \textbf{Visitante}: se trata de un navegante que accede a la \textit{web}. En este caso, las funcionalidades que tiene disponibles son más limitadas, ya que únicamente se le permite analizar URLs y visualizar los resultados.
	\item \textbf{Usuario registrado}: un usuario registrado (rol <<estándar>>) tiene acceso a funcionalidades colaborativas. Es decir, además de poder escanear URLS, tiene permiso para reportar enlaces que sepa que son legítimos o \textit{phishing}. Estos serán revisados por administradores y etiquetados (de forma que pertenezcan a una lista blanca o a una lista negra). Además, también podrán notificar análisis incorrectos. De esta forma, si un usuario piensa que una URL es, por ejemplo, \textit{phishing}, y los modelos predicen que se trata de un enlace legítimo, puede notificarlo. Este aviso llegará a los administradores, quienes podrán examinar la instancia e incluirla en el \textit{dataset} para ser estudiada por modelos futuros y mejorar el rendimiento de la aplicación.
	\item \textbf{Administrador}: los usuarios que tengan este rol tendrán acceso a la funcionalidad completa de la \textit{web}. Podrán crear nuevos modelos de aprendizaje y entrenarlos con las instancias que consideren, además de editar los existentes. También podrán revisar todas las notificaciones y reportes de usuarios de la aplicación e inluir nuevas instancias en el \textit{dataset}, además de modificar las existentes.
\end{itemize}

\section{Catálogo de historias de usuario}
\label{s:hu}

\begin{table}
	\scalebox{0.80}{
	\begin{tabular}{@{}p{3em} p{6em} p{8em} p{8em} p{10em}@{}}		
	\toprule
	\textbf{ID} & \textbf{Como} & \textbf{Quiero} & \textbf{Para} & \textbf{Aceptación}\\
	\midrule
		HU-1 & Visitante & Analizar URLs & Saber si un enlace es fraudulento (\textit{phishing}) o legítimo & La página debe mostrar el resultado tras el análisis.
		
		La \textit{web} debe informar si no es posible mediante un mensaje ilustrativo.\\\\
		HU-2 & Visitante & Poder registrarme & Crear una cuenta en la aplicación & La cuenta es persistente y se puede acceder a ella. \\\\
		HU-3 & Visitante & Poder iniciar sesión & Acceder a más funcionalidades & Al introducir credenciales correctas se accede a una cuenta y se desbloquean opciones colaborativas. \\\\
		HU-4 & Usuario registrado & Reportar URLs & Que sean incluídas en una lista blanca o negra & Existe un lugar donde reportar URLs cuando se está seguro de la clase a la que pertenecen.\\\\
		HU-5 & Usuario registrado & Notificar análisis incorrectos & Que las instancias sean revisadas y poder mejorar los algoritmos de ML & Existe un botón con el que se reporta automáticamente que se considera que un análisis ha fallado.\\\\
		HU-6 & Administrador & Administrar instancias (URLs) & Poder mantener un dataset actualizado & Se pueden añadir instancias.
		
		Se pueden generar ficheros \texttt{csv} con las instancias disponibles.
		
		Se pueden modificar instancias.
		
		Se pueden consultar las instancias reportadas por los usuarios.\\\\
		HU-7 & Administrador & Administrar modelos de \textit{machine learning} & Que estén disponibles en la página y puedan ser utilizados por los usuarios para realizar análisis & Se pueden crear nuevos modelos desde la web.
		
			Se pueden probar modelos existentes.
			
			Se pueden editar modelos existentes.
		\\
	\bottomrule
	\end{tabular}
	}
	\caption[Historias de usuario: entrevista con el \textit{product ownew}]{Historias de usuario recogidas durante las entrevistas con el \textit{product owner}}
	\label{hu:estructura-tabular}
\end{table}

A modo ilustrativo y con el fin de realizar una memoria lo más completa posible (coherente con las metodologías ágiles), se facilitan en la tabla~\ref{hu:estructura-tabular} algunas historias de usuario tomadas durante las entrevistas con el \textit{product owner}. Sin embargo, el catálogo completo de requisitos se encuentra en el apartado~\ref{s:cat-requisitos}.


\section{Catálogo de requisitos}
\label{s:cat-requisitos}

\section{Prototipado}
\label{s:mockups}

Como es conocido, en el mundo del desarrollo \textit{software} suele haber una diferencia en el entendimiento de una aplicación por parte del cliente y del equipo de desarrollo que puede desembocar en malentendidos que causen retrasos temporales y pérdidas económicas.

Con el fin de reducir dichas desventajas y realizar un diseño que se adapte a los requerimientos del usuario, se ha decidido realizar una fase de prototipado durante las entrevistas del producto. Esto ha permitido identificar posibles diferencias entre el equipo del desarrollo y el cliente (representado por el \textit{product owner}), facilitando la comunicación y colaboración entre ambas partes.

Se adjuntan a continuación los diferentes \textit{mockups} del analizador de \textit{phishing} realizados durante esta fase.


\begin{figure}[h]
	\caption{Prototipos: página principal}
	\centering
	\includegraphics[width=\textwidth]{../img/anexos/mockups/1-mockups-index}
	\label{mock:index}
\end{figure}

\begin{figure}[h]
	\caption{Prototipos: \textit{dashboard}}
	\centering
	\includegraphics[width=\textwidth]{../img/anexos/mockups/2-mockups-dashboard}
	\label{mock:dashboard}
\end{figure}

\begin{figure}[h]
	\caption{Prototipos: reportar una URL}
	\centering
	\includegraphics[width=\textwidth]{../img/anexos/mockups/3-mockups-report_url}
	\label{mock:report_url}
\end{figure}

\begin{figure}[h]
	\caption{Prototipos: perfil del usuario}
	\centering
	\includegraphics[width=\textwidth]{../img/anexos/mockups/4-mockups-profile}
	\label{mock:profile}
\end{figure}

\begin{figure}[h]
	\caption{Prototipos: administración de modelos}
	\centering
	\includegraphics[width=\textwidth]{../img/anexos/mockups/5-mockups-models}
	\label{mock:model-admin}
\end{figure}

\begin{figure}[h]
	\caption{Prototipos: creación de modelos}
	\centering
	\includegraphics[width=\textwidth]{../img/anexos/mockups/6-mockups-new_model}
	\label{mock:model-new}
\end{figure}

\begin{figure}[h]
	\caption{Prototipos: evaluación de modelos}
	\centering
	\includegraphics[width=\textwidth]{../img/anexos/mockups/7-mockups-test_model}
	\label{mock:model-test}
\end{figure}

\begin{figure}[h]
	\caption{Prototipos: selección de instancias}
	\centering
	\includegraphics[width=\textwidth]{../img/anexos/mockups/8-mockups-select_instances}
	\label{mock:instance-selection}
\end{figure}

\begin{figure}[h]
	\caption{Prototipos: administración de instancias}
	\centering
	\includegraphics[width=\textwidth]{../img/anexos/mockups/9-mockups-instances}
	\label{mock:instance-admin}
\end{figure}

\begin{figure}[h]
	\caption{Prototipos: edición de instancias}
	\centering
	\includegraphics[width=\textwidth]{../img/anexos/mockups/10-mockups-edit_instance}
	\label{mock:instances-edit}
\end{figure}


\section{Especificación de requisitos}
\label{s:requisitos}

Se detalla a continuación la especificación de requisitos tradicional de la aplicación \textit{web} propuesta basada en el catálogo de requisitos de la sección~\ref{s:cat-requisitos}.

\subsection{Diagrama de casos de uso}
\label{ss:diagrama-casos-uso}

Partiendo de las historias de usuario extraídas en las entrevistas realizadas con el \textit{product owner} y de los \textit{mockups} mostrados en la sección~\ref{s:mockups}, se ha extraído el diagrama de casos de uso disponible en la figura~\ref{b:diagrama-cu}.

\begin{figure}[h]
	\caption{Diagramas: casos de uso}
	\centering
	\includegraphics[width=\textwidth]{../img/anexos/diagrams/cu}
	\label{b:diagrama-cu}
\end{figure}

A continuación se muestra la tabla correspondiente a cada caso de uso.

% Caso de Uso 1 -> Escanear URL
\begin{table}[p]
	\centering
	\begin{tabularx}{\linewidth}{ p{0.21\columnwidth} p{0.71\columnwidth} }
		\toprule
		\textbf{CU-1}    & \textbf{Escanear URL}\\
		\toprule
		\textbf{Versión}              & 1.0    \\
		\textbf{Autor}                & Patricia Hernando Fernández \\
		\textbf{Requisitos asociados} & RF-xx, RF-xx \\
		\textbf{Descripción}          & Permitir que un usuario realice un escaneo de la URL que desee. Para ello, seleccionará los modelos que considere adecuados y se mostrarán los resultados en un \textit{dashboard} (CU-2 en la tabla~\ref{cu:visualizar-resultados}). \\
		\textbf{Precondición}         & No hay precondiciones. \\
		\textbf{Acciones}             &
		\begin{enumerate}
			\def\labelenumi{\arabic{enumi}.}
			\tightlist
			\item El usuario introduce la URL.
			\item El usuario selecciona los modelos de ML que considere (consultar CU-1.1 en la tabla~\ref{cu:seleccionar-modelos-ml}).
			\item El usuario decide si quiere realizar un análisis rápido.
			\item El usuario pulsa el botón de <<realizar análisis>>.
		\end{enumerate}\\
		\textbf{Postcondición}        & No hay postcondiciones. \\
		\textbf{Excepciones}          & En caso de que la URL introducida sea inalcanzable (tras aplicar reintentos) o que no haya ningún modelo disponible, el usuario será notificado y redirigido a la página principal. \\
		\textbf{Importancia}          & Alta \\
		\bottomrule
	\end{tabularx}
	\caption{CU-1 Escanear URL.}
	\label{cu:escanear-url}
\end{table}


% Caso de Uso 1.1 -> Seleccionar modelos
\begin{table}[p]
	\centering
	\begin{tabularx}{\linewidth}{ p{0.21\columnwidth} p{0.71\columnwidth} }
		\toprule
		\textbf{CU-1.1}    & \textbf{Seleccionar modelos con los que analizar}\\
		\toprule
		\textbf{Versión}              & 2.0    \\
		\textbf{Autor}                & Patricia Hernando Fernández \\
		\textbf{Requisitos asociados} & RF-xx, RF-xx \\
		\textbf{Descripción}          & Se mostrará al usuario los distintos modelos disponibles (y visibles) para que elija los que desee. \\
		\textbf{Precondición}         & Estar realizando el CU-1 (disponible en la tabla~\ref{cu:escanear-url}). \\
		\textbf{Acciones}             &
		\begin{enumerate}
			\def\labelenumi{\arabic{enumi}.}
			\tightlist
			\item Se cargan los modelos disponibles en la base de datos.
			\item El usuario selecciona los modelos que quiera utilizar.
		\end{enumerate}\\
		\textbf{Postcondición}        & Los modelos seleccionados quedan guardados. \\
		\textbf{Excepciones}          & En caso de no haber modelos disponibles la lista quedará vacía y se controlará la excepción en el CU-1 (tabla~\ref{cu:escanear-url}). \\
		\textbf{Importancia}          & Baja \\
		\bottomrule
	\end{tabularx}
	\caption{CU-1.1 Seleccionar modelos con los que analizar.}
	\label{cu:seleccionar-modelos-ml}
\end{table}

% Caso de Uso 2 -> Visualizar resultados
\begin{table}[p]
	\centering
	\begin{tabularx}{\linewidth}{ p{0.21\columnwidth} p{0.71\columnwidth} }
		\toprule
		\textbf{CU-2}    & \textbf{Visualizar resultados}\\
		\toprule
		\textbf{Versión}              & 1.0    \\
		\textbf{Autor}                & Patricia Hernando Fernández \\
		\textbf{Requisitos asociados} & RF-xx, RF-xx \\
		\textbf{Descripción}          & Se muestra al usuario los resultados tras haber analizado la URL mediante los algoritmos de ML seleccionados. Para ello se hará uso de un \textit{dashboard}, donde además se podrá notificar un falso resultado (CU-2.1 en la tabla~\ref{cu:notificar-falso-resultado}).\\
		\textbf{Precondición}         & Haber realizado un análisis previamente (CU-1 en la tabla~\ref{cu:escanear-url}). \\
		\textbf{Acciones}             &
		\begin{enumerate}
			\def\labelenumi{\arabic{enumi}.}
			\tightlist
			\item El usuario es redirigido a un \textit{dashboard}.
			\item El usuario puede interactuar con los distintos gráficos y la información mostrada en la página.
			\item El usuario puede notificar si considera que el análisis es erróneo (CU-2.1 en la tabla~\ref{cu:notificar-falso-resultado}).
		\end{enumerate}\\
		\textbf{Postcondición}        & No hay postcondiciones \\
		\textbf{Excepciones}          & En caso de intentar acceder al \textit{dashboard} sin haber realizado un análisis previo, el usuario será redirigido a la página principal con un mensaje informativo.\\
		\textbf{Importancia}          & Alta \\
		\bottomrule
	\end{tabularx}
	\caption{CU-2 Visualizar resultados.}
	\label{cu:visualizar-resultados}
\end{table}


% Caso de Uso 2.1 -> Notificar falso resultado
\begin{table}[p]
	\centering
	\begin{tabularx}{\linewidth}{ p{0.21\columnwidth} p{0.71\columnwidth} }
		\toprule
		\textbf{CU-2.1}    & \textbf{Notificar falso resultado}\\
		\toprule
		\textbf{Versión}              & 1.0    \\
		\textbf{Autor}                & Patricia Hernando Fernández \\
		\textbf{Requisitos asociados} & RF-xx, RF-xx \\
		\textbf{Descripción}          & Permite a un usuario notificar automáticamente a la aplicación si considera que el resultado de un análisis es erróneo.\\
		\textbf{Precondición}         & Encontrarse en el \textit{dashboard} tras un análisis (CU-2 en la tabla~\ref{cu:visualizar-resultados}) y haber iniciado sesión (CU-4 en la tabla~\ref{cu:iniciar-sesion}).\\
		\textbf{Acciones}             &
		\begin{enumerate}
			\def\labelenumi{\arabic{enumi}.}
			\tightlist
			\item El usuario pulsa el botón <<notificar falso resultado>>.
		\end{enumerate}\\
		\textbf{Postcondición}        & Se registra la URL notificada en la base de datos con el campo <<sugerencia>> establecido al valor opuesto del resultado mostrado. \\
		\textbf{Excepciones}          & Si el usuario no ha iniciado sesión, será informado de que la notificación no ha sido realizada por este motivo. En caso de errores con la base de datos se controlarán y se informará al usuario de la notificación no realizada.\\
		\textbf{Importancia}          & Baja \\
		\bottomrule
	\end{tabularx}
	\caption{CU-2.1  Notificar falso resultado.}
	\label{cu:notificar-falso-resultado}
\end{table}


% Caso de Uso 3 -> Registrarse
\begin{table}[p]
	\centering
	\begin{tabularx}{\linewidth}{ p{0.21\columnwidth} p{0.71\columnwidth} }
		\toprule
		\textbf{CU-3}    & \textbf{Registrarse}\\
		\toprule
		\textbf{Versión}              & 1.0    \\
		\textbf{Autor}                & Patricia Hernando Fernández \\
		\textbf{Requisitos asociados} & RF-xx, RF-xx \\
		\textbf{Descripción}          & Permite a un usuario crear una cuenta en la aplicación.\\
		\textbf{Precondición}         & El usuario no debe estar registrado ni haber iniciado sesión (CU-4 en la tabla~\ref{cu:iniciar-sesion}). \\
		\textbf{Acciones}             &
		\begin{enumerate}
			\def\labelenumi{\arabic{enumi}.}
			\tightlist
			\item El usuario selecciona <<Registrarse>>.
			\item El usuario introduce el nombre de usuario que considere.
			\item El usuario introduce un \textit{email} asociado a la cuenta.
			\item El usuario introduce la contraseña deseada.
			\item El usuario pulsa el botón de <<Registrar>>.
		\end{enumerate}\\
		\textbf{Postcondición}        & La información del nuevo usuario queda almacenada en la base de datos. \\
		\textbf{Excepciones}          & En caso de que los datos estén repetidos o el \textit{email} sea incorrecto, se pedirá al usuario que corrija los errores y no se creará la nueva cuenta. Si hay un usuario \textit{logueado} será redirigido a la página principal con un mensaje informativo.\\
		\textbf{Importancia}          & Media \\
		\bottomrule
	\end{tabularx}
	\caption{CU-3 Registrarse.}
	\label{cu:registrarse}
\end{table}


% Caso de Uso 4 -> Iniciar sesión
\begin{table}[p]
	\centering
	\begin{tabularx}{\linewidth}{ p{0.21\columnwidth} p{0.71\columnwidth} }
		\toprule
		\textbf{CU-4}    & \textbf{Iniciar sesión}\\
		\toprule
		\textbf{Versión}              & 1.0    \\
		\textbf{Autor}                & Patricia Hernando Fernández \\
		\textbf{Requisitos asociados} & RF-xx, RF-xx \\
		\textbf{Descripción}          & Permite a un usuario iniciar sesión en la aplicación.\\
		\textbf{Precondición}         & El usuario no debe haber iniciado sesión en el navegador (CU-4 en la tabla~\ref{cu:iniciar-sesion}). \\
		\textbf{Acciones}             &
		\begin{enumerate}
			\def\labelenumi{\arabic{enumi}.}
			\tightlist
			\item El usuario pulsa el botón de <<Iniciar sesión>>.
			\item El usuario introduce el nombre de usuario asociado a su cuenta.
			\item El usuario introduce la contraseña asociada a la cuenta.
			\item El usuario pulsa el botón de <<Iniciar sesión>>.
		\end{enumerate}\\
		\textbf{Postcondición}        & La información del usuario queda cargada en las variables de sesión. \\
		\textbf{Excepciones}          & En caso de que los datos sean incorrectos el usuario será notificado. Si hay un usuario \textit{logueado} será redirigido a la página principal con un mensaje informativo.\\
		\textbf{Importancia}          & Media \\
		\bottomrule
	\end{tabularx}
	\caption{CU-3 Iniciar sesión.}
	\label{cu:iniciar-sesion}
\end{table}

% Caso de Uso 5 -> Reportar URL
\begin{table}[p]
	\centering
	\begin{tabularx}{\linewidth}{ p{0.21\columnwidth} p{0.71\columnwidth} }
		\toprule
		\textbf{CU-5}    & \textbf{Reportar URL}\\
		\toprule
		\textbf{Versión}              & 1.0    \\
		\textbf{Autor}                & Patricia Hernando Fernández \\
		\textbf{Requisitos asociados} & RF-xx, RF-xx \\
		\textbf{Descripción}          & Permite a un usuario reportar una
		URL si considera que pertenece a una lista blanca o a una lista negra.\\
		\textbf{Precondición}         & El usuario debe de haber iniciado sesión (CU-4 en la tabla~\ref{cu:iniciar-sesion}). \\
		\textbf{Acciones}             &
		\begin{enumerate}
			\def\labelenumi{\arabic{enumi}.}
			\tightlist
			\item El usuario introduce la URL a reportar.
			\item El usuario selecciona el tipo de lista a la que pertenece la URL.
			\item El usuario pulsa el botón de <<Enviar>>.
		\end{enumerate}\\
		\textbf{Postcondición}        & Se registra la URL reportada en la base de datos con el campo <<sugerencia>> establecido al valor seleccionado en el desplegable (ejemplo: \textit{white-list}). \\
		\textbf{Excepciones}          & En caso de no haber iniciado sesión, el usuario será redirigido a una pantalla especial (error 403, acceso restringido) con un enlace para iniciar sesión. Se controla en aplicación que el usuario haya rellenado todos los campos necesarios. \\
		\textbf{Importancia}          & Baja \\
		\bottomrule
	\end{tabularx}
	\caption{CU-5 Reportar URL.}
	\label{cu:reportar-url}
\end{table}


