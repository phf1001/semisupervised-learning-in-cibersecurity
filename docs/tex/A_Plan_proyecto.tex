\apendice{Plan de Proyecto Software}

\section{Introducción}

\section{Planificación temporal}

\subsection{Planificación por \textit{sprints}}


\subsubsection{\textit{Sprint 1:}}

\begin{itemize}
	\item \textbf{\textit{Planning meeting}}
	
	Durante la reunión se marcaron los siguientes objetivos:
	
	\begin{enumerate}
		\item Configuración básica: incluyendo la creación del repositorio, la correcta instalación de ZenHub, la creación de entornos virtuales (miniconda, SKLearn, etc.) y la familiarización con conceptos \textit{scrum}: \textit{milestones, sprints, epics}, etc.
		\item Memoria: comienzo de la redacción incluyendo las secciones de introducción, conceptos teóricos (aprendizaje automático) y trabajo relacionado.
		\item Investigación: búsqueda del código SSADR-CoF y de las bases de datos utilizadas en el paper.
		\item Lectura de papers: Engelen \& Hoos~\cite{engelen2020surveyOnSemiSupervised}, García, Triguero \& Herrera~\cite{triguero2015SelflabeledTechniques}, y Zhou \& Duan~\cite{zhou2021SemisupervisedRecommendationAttack}.
	\end{enumerate}
	
	\item \textbf{Marcas temporales}
	El \textit{sprint} se desarrolló entre el 24 de septiembre de 2022 y el 2 de octubre del 2022.
	
	\item \textbf{\textit{Burndown Report}}
	\begin{figure}[h]
		\caption{\textit{Burndown Report Sprint 01}}
		\centering
		\includegraphics[width=\textwidth]{../img/anexos/s01_bdr}
	\end{figure}
	
	Como se puede comprobar, no todos los objetivos marcados fueron cumplidos: la estimación del tiempo fue demasiado optimista, además de no contar con el tiempo requerido en solucionar problemas técnicos (\LaTeX{}). Se dejó para próximos sprints la lectura del último paper.

	\item \textbf{\textit{Sprint review meeting}}
	Durante la reunión se fijaron ciertas correcciones en la memoria (mejorar referencias bibliográficas y la sección de <<Trabajos relacionados>>), además de la necesidad de introducir una sección teórica de ataques a los sistemas de recomendación.
	
\end{itemize}


\subsubsection{\textit{Sprint 2:}}
\begin{itemize}
	\item \textbf{\textit{Planning meeting}}
	Objetivos del siguiente Sprint:
	
	\begin{enumerate}
		\item Configuración: debido a la gran cantidad de tiempo invertida en solucionar errores de compilación en \LaTeX{}, se decidió migrar el proyecto a una nueva instalación basada en Debian.
		\item Correcciones: aspectos estilísticos y completar información.
		\item Lectura: Mingdan y Qingshan~\cite{mingdan2018ShillingAttacksAReview} con el objetivo de introducir una sección teórica de ataques.
		\item Memoria: redacción completa de los modelos de ataque en los aspectos teóricos.
		
	\end{enumerate}
	
	\item \textbf{Marcas temporales}
	El \textit{sprint} se desarrolló entre el 3 de octubre de 2022 y el 18 de octubre del 2022.
	
	\item \textbf{\textit{Burndown Report}}
	\begin{figure}[h]
		\caption{\textit{Burndown Report Sprint 02}}
		\centering
		\includegraphics[width=\textwidth]{../img/anexos/s02_bdr}
	\end{figure}
		
	En este \textit{sprint} sí se cumplió con los objetivos marcados. Sin embargo, la estimación de tiempo tampoco fue la adecuada, requiriendo más de lo previsto.
	\item \textbf{\textit{Sprint review meeting}}
	Durante la reunión se resolvieron dudas acerca de bibliografía, referencias y trabajo previo. Además, se acordó empezar a programar, definiendo así los \textit{issues} desarrollados en el siguiente \textit{sprint}.
	
\end{itemize}

\subsubsection{\textit{Sprint 3:}}
\begin{itemize}
	\item \textbf{\textit{Planning meeting}}
	
	Durante esta reunión, se decidió empezar a programar el \textit{co-forest}. Para ello, se definieron los siguientes pasos:
	
	\begin{enumerate}
		
		\item Librerías: se acordó aprender a utilizar las librerías más comunes en el \textit{data science}. Entre ellas: MatplotLib y SKLearn. Además, se requirió la correcta configuración del entorno virtual, haciendo que el tiempo dedicado al \textit{issue} fuese mayor de lo estimado (problemas en el \texttt{PATH} y con las dependencias).
		
		\item \textit{SKLearn}: aprovechando la correcta documentación de la librería, se decidió repasar los conceptos teóricos básicos, además del manejo de la <<interfaz>> (métodos comunes). Entre ellos:
		
		\begin{itemize}
			\item \textit{Decision trees}
			\item \textit{Self training}
			\item \textit{Random Forest}
		\end{itemize}
	
		\item Lectura: se concertó la relectura del artículo de Zhou~\cite{zhou2021SemisupervisedRecommendationAttack} con la intención de comprender el algoritmo y del \textit{paper} <<original>> del \textit{co-forest}~\cite{originalCoForest2007}. Durante el proceso de programación, además, se encontró la tesis de Van Engelen~\cite{engelen2018thesis} y se añadió al conjunto.
		\item Documentación: se acordó la corrección de los errores previamente señalados y la inclusión del \textit{sprint} en los anexos.
		\item Programación del \textit{Co-Forest}: se programó el pseudocódigo ilustrado en la Tesis de Van Engelen~\cite{engelen2018thesis}, que es muy similar al original~\cite{originalCoForest2007} pero con algunas diferencias. Inicialmente se intentó usar el \textit{Random Forest} de \textit{SKLearn}, pero se descartó la idea debido a la poca versatilidad que se ofrecía para manejar los \textit{concomitant ensembles}. Se han de corregir ciertos factores, pero se pospondrá hasta la correcta discusión con el tutor.
		
	\end{enumerate}
	
	\item \textbf{Marcas temporales}	
	El \textit{sprint} se desarrolló entre el 19 de octubre de 2022 y el 2 de noviembre del 2022.
	
	\item \textbf{\textit{Burndown Report}}
	
		\begin{figure}[h]
		\caption{\textit{Burndown Report Sprint 03}}
		\centering
		\includegraphics[width=\textwidth]{../img/anexos/s03_bdr}
		\end{figure}
	
	En este \textit{sprint} se cumplió con los objetivos marcados. Nuevamente, la estimación del tiempo fue inferior a la real (se pensaba que se podría depender más de librerías existentes de lo que se pudo en realidad), calculándose un total de aproximadamente 25 horas reales.

	\item \textbf{\textit{Sprint review meeting}}
	
	Durante la revisión del \textit{sprint}, se llegó a la conclusión de que el pseudocódigo podía ser mejor implementado aprovechando ciertas librerías de \textit{Python}. Se comentó cómo mejorar complejidades espaciales y reducir el código. Se fijaron objetivos para las próximas semanas.
	
\end{itemize}


\subsubsection{\textit{Sprint 4:}}
\begin{itemize}
	\item \textbf{\textit{Planning meeting}}
	
	Durante la reunión se acordaron los siguientes objetivos:
	
	\begin{enumerate}
		\item Reimplementación del código: se acordó volver a programar el \textit{co-forest}, esta vez implementando una versión más <<\textit{pythoniana}>> con el fin de mejorar la complejidad espacial y facilitar la lectura.
		\item Curso de Numpy: se decidió que sería interesante la realización de un curso para aprender a utilizar la librería y aplicarla al código.
		\item Curso de Pandas: aprovechando la relación con el punto anterior, se acordó completar también un curso de esta librería.
		\item Memoria: corregir aspectos anteriores e incluir toda la teoría relacionada con el \textit{co-forest}.
	\end{enumerate}
	
	\item \textbf{Marcas temporales}
	El \textit{sprint} se desarrolló entre el 3 de noviembre de 2022 y el 15 de noviembre del 2022.
			
	\item \textbf{\textit{Burndown Report}}
	\begin{figure}[h]
		\caption{\textit{Burndown Report Sprint 04}}
		\centering
		\includegraphics[width=\textwidth]{../img/anexos/s04_bdr}
	\end{figure}
	
	En este \textit{sprint} se completaron los objetivos, aunque quedaron pendientes ciertos aspectos a comentar respecto al código. Destacar que, debido a que se terminaron antes de lo planeado los \textit{issues} planificados, se aprovechó para modificar ciertos aspectos pendientes relacionados con la memoria y para probar correctamente el código. Esto hizo que el tiempo real dedicado haya sido ligeramente superior al estimado (más tiempo de documentación).
	
	\item \textbf{\textit{Sprint review meeting}}
	
\end{itemize}





\subsubsection{\textit{Sprint N:}}
\begin{itemize}
	\item \textbf{\textit{Planning meeting}}
	\item \textbf{Marcas temporales}		
	\item \textbf{\textit{Burndown Report}}
	\item \textbf{\textit{Sprint review meeting}}
\end{itemize}

\section{Estudio de viabilidad}

\subsection{Viabilidad económica}

\subsection{Viabilidad legal}


