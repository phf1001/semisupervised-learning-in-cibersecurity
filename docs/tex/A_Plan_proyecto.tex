\apendice{Plan de Proyecto Software}

\section{Introducción}

\section{Planificación temporal}

\subsection{Planificación por \textit{sprints}}

\subsubsection{\textit{Sprint 1: Sprint inicial}}

\begin{itemize}
	\item \textbf{\textit{Planning meeting}}
	
	Dentro de la reunión se marcaron los siguientes objetivos:
	
	\begin{enumerate}
		\item Configuración básica: incluyendo la creación del repositorio, la correcta instalación de ZenHub, la creación de entornos virtuales (miniconda, SKLearn, etc.) y la familiarización con conceptos \textit{scrum}: milestones, sprints, epics, etc.
		
		\item Memoria: comienzo de la redacción incluyendo las secciones de introducción, conceptos teóricos (aprendizaje automático) y trabajo relacionado.
		
		\item Investigación: búsqueda del código SSADR-CoF y de las bases de datos utilizadas en el paper.
		
		\item Lectura de papers: 
	\end{enumerate}
	
	\item \textbf{Marcas temporales}
	
	El sprint se desarrolló entre el 24 de septiembre de 2022 y el 2 de octubre del 2022.
	
	\item \textbf{\textit{Burndown chart}}
	
	Como se puede comprobar, no todos los objetivos marcados fueron cumplidos: la estimación del tiempo fue demasiado optimista. Se dejó por terminar la lectura del último paper.
	
	\item \textbf{\textit{Sprint review meeting}}
	Durante la reunión se fijaron ciertas correcciones en la memoria.
	
	
\end{itemize}


\subsubsection{\textit{Sprint 2:}}

\begin{itemize}
	\item \textbf{\textit{Planning meeting}}
	
	Objetivos del siguiente Sprint:
	
		\begin{enumerate}
			
		\item Configuración: debido a la gran cantidad de tiempo invertida en solucionar errores de compilación en Latex, se decidió migrar el proyecto a una nueva instalación basada en Debian.
		\item Correcciones: aspectos estilísticos y completar información.
		
		\item Lectura: 
		\item Memoria: finalización de los aspectos teóricos iniciales: inclusión total de los modelos de ataque.
		
	\end{enumerate}

	\item \textbf{Marcas temporales}
	
El sprint se desarrolló entre el 03 de octubre de 2022 y el 18 de octubre del 2022.
			
	\item \textbf{\textit{Burndown chart}}
	\item \textbf{\textit{Sprint review meeting}}
\end{itemize}





\subsubsection{\textit{Sprint N:}}
\begin{itemize}
	\item \textbf{\textit{Planning meeting}}
	\item \textbf{Marcas temporales}		
	\item \textbf{\textit{Burndown chart}}
	\item \textbf{\textit{Sprint review meeting}}
\end{itemize}


\section{Estudio de viabilidad}

\subsection{Viabilidad económica}

\subsection{Viabilidad legal}


