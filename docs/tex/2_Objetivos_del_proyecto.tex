\capitulo{2}{Objetivos del proyecto}

Previo a la introducción de objetivos técnicos y de desarrollo, se facilita al lector el contexto en el que se desarrolla este proyecto de investigación con el fin de que se familiarice con el mismo.

\section{Contexto}
\label{contexto}

Es reseñable que este proyecto se sostiene sobre tres pilares complementarios pero a su vez independientes.

\subsection{Aprendizaje semisupervisado}

Por un lado, en este trabajo se pretende ahondar en el concepto de aprendizaje semisupervisado. Para ello, se van a implementar y validar diversos algoritmos de este tipo, en concreto el \textit{co-forest} (consultar aspectos teóricos en la sección~\ref{coforest-teoria} y la experimentación en la sección~\ref{coforest-exp}), el \textit{democratic-co learning} (secciones~\ref{democraticco-teoria} y~\ref{democraticco-exp}) y el \textit{tri-training} (secciones~\ref{tritraining-teoria} y~\ref{tritraining-exp}).

Es reseñable que se pretende alcanzar un nivel de comprensión superior a la mera programación, realizando incluso pequeñas aportaciones a los algoritmos (basadas en la experimentación) en aquellos casos donde el pseudocódigo facilitado por los autores carezca de detalles de implementación.

\subsection{Ciberseguridad}

Implementar algoritmos es un ejercicio relevante, pero carece de sentido si no pueden ser utilizados. Por ello, se pretende encontrar una aplicación a los mismos. Concretamente, se ha propuesto el modelado de un problema real relacionado con la ciberseguridad.

En un primer lugar se valoró estudiar la detección de ataques en sistemas de recomendación~\cite{zhou2021SemisupervisedRecommendationAttack}, pero se desestimó por su bajo desempeño. Finalmente, se ha escogido la detección de \textit{phishing}~\cite{featuresPhishing2018Gupta}.

Es destacable que previo a la selección de las ramas mencionadas anteriormente, se realizó una investigación en la literatura de la materia leyendo y comparando diversos tópicos como intrusión en redes, localización de \textit{markets} ilegales en la \textit{deep web}, detección de \textit{malware} o descubrimiento de dispositivos anómalos en redes IoT.

\subsection{Desarrollo y despliegue}

Por último, se pretende diseñar e implementar una herramienta (\textit{web}) que permita poner al servicio de la comunidad el conocimiento desarrollado. Se ha decidido crear un analizador de \textit{phishing} en formato página \textit{web}, permitiendo además la administración avanzada de modelos de \textit{machine learning} e instancias de aprendizaje. Además, se ha desplegado dicha herramienta utilizando Heroku\footnote{Disponible en \url{https://krini.herokuapp.com/index}.} y Docker\footnote{\textit{Scripts} e instrucciones de despliegue disponibles en \url{https://github.com/phf1001/semisupervised-learning-in-cibersecurity/tree/main/docker-deploy-kit}.}.

\section{Objetivos}

Entendidos como las metas a alcanzar durante el desarrollo del proyecto, se pueden clasificar en dos secciones.

\subsection{Objetivos técnicos}
\label{Objetivos técnicos}

Se han propuesto los siguientes objetivos técnicos:

\begin{enumerate}
	\item \textbf{Implementar y evaluar distintos algoritmos de aprendizaje semisupervisado:} con rigor científico y procurando una codificación eficiente.
	\item \textbf{Experimentar con detalles de implementación no concretados:} para comprobar cómo determinadas decisiones pueden afectar el desempeño de un algoritmo.
	\item \textbf{Comparar contra implementaciones de terceros:} logrando unas condiciones de análisis justas.
	\item \textbf{Modelar un problema de ciberseguridad:} ofreciendo una solución práctica basada en el aprendizaje predictivo.
	\item \textbf{Establecer medidas de seguridad:} siendo consciente de que los métodos de extracción pueden amenazar vulnerabilidades y mitigando el riesgo.
	\item \textbf{Evaluar la calidad de la solución aportada:} en términos de rendimiento.
	\item \textbf{Aprendizaje de nuevas tecnologías:} y familiarización con la programación \textit{web}.
	
\end{enumerate}

\subsection{Objetivos de desarrollo \textit{software}}
\label{Objetivos de desarrollo}

Se han propuesto los siguientes objetivos de desarrollo software:

\begin{enumerate}
	\item \textbf{Implementar de forma estándar los algoritmos:} siguiendo interfaces conocidas por la comunidad (métodos \texttt{fit} y \texttt{predict}).
	\item \textbf{Adoptar metodologías ágiles de desarrollo} y familiarización con el flujo de Scrum.
	\item \textbf{Documentar el producto desarrollado:} de forma concisa y que resuma, utilizando herramientas visuales (como diagramas) y multimedia (como tutoriales), la arquitectura planteada y su utilización.
	\item \textbf{Lograr un código de calidad:} desarrollando un producto que tenga una deuda técnica baja y una curva de aprendizaje asumible mediante el uso de herramientas como SonarCloud, DeepSource o Travis CI.
	\item \textbf{Aplicar la integración y entrega continua:} de forma que se garantice el desarrollo de un producto mínimo viable que permita la detección temprana de errores y facilite la corrección por parte del tutor.
	\item \textbf{Vincular fuentes de persistencia:} como bases de datos SQL.
	\item \textbf{Desplegar en plataformas:} logrando una familiarización con nuevas tecnologías en auge como Heroku y Docker.
	\item \textbf{Automatizar pruebas:} en distintas fases del ciclo de vida del producto mediante el uso de herramientas como TravisCI y Selenium.
	
\end{enumerate}