\capitulo{1}{Introducción}

A diferencia de unas décadas atrás, la sociedad actual está gobernada por los datos. La transición a la era de la información puede ser compleja para determinados colectivos y, consecuentemente, han surgido grupos que pretenden facilitar la experiencia de los usuarios, mientras que otros buscan aprovechar la falta de información general para lograr un beneficio.

Este trabajo de investigación pretende explorar cómo la minería de datos, en concreto el aprendizaje semisupervisado, puede aplicarse a la seguridad informática con el fin de lograr un entorno digital de confianza. Para ello, se han abierto dos ramas de investigación: la detección automática de ataques en sistemas de recomendación y la detección de \textit{phishing.}

Un sistema de recomendación es una herramienta desarrollada con el fin de resumir información y facilitar la toma de decisiones. Son sistemas que pretenden realizar sugerencias de objetos que pueden resultar interesantes para un determinado perfil. Económicamente, este tipo de algoritmo es un claro objeto de interés, puesto que puede influir en la toma de decisiones de los compradores y hacer que se inclinen por un determinado producto (por ejemplo, el que tenga una mejor valoración). Los atacantes conocen esta situación y manipulan estas herramientas mediante el uso de perfiles falsos con el fin de beneficiar sus productos o perjudicar los de la competencia. Por ello, uno de los objetivos del proyecto es explorar cómo el aprendizaje semisupervisado puede ayudar a detectar los ataques a sistemas de recomendación, diferenciando entre perfiles genuinos e inyectados, además de comprobar la veracidad de los trabajos previos planteados por otros investigadores.

Por otro lado, se denomina \textit{phishing} a la técnica de ingeniería social en la que un atacante suplanta la identidad de una entidad de confianza con el fin de extraer datos o información personal de un determinado usuario. Este tipo de ataque es uno de los más comunes y peligrosos. Según el informe \textit{State of the Phish Report}, en el 2021, un 83\% de las empresas encuestadas sufrieron un ataque de estas características y el 54\% acabó con brechas de datos de clientes~\cite{phishingPorcentajeExito}. Por ello, se pretende crear un sistema de detección que permita distinguir enlaces falsos de auténticos mediante ténicas de aprendizaje semisupervisado.

Para complementar la investigación realizada, se ha desarrollado y desplegado (en Heroku y en local mediante Docker) una herramienta \textit{web} denominada Krini\footnote{Disponible en \url{https://krini.herokuapp.com/index}.}, que consiste en un analizador de páginas que diferencia entre sitios fraudulentos y legítimos, además de permitir crear modelos, gestionar instancias y aceptar sugerencias de usuarios desde la aplicación.

Es destacable que todos los algoritmos de aprendizaje semisupervisado utilizados han sido implementados y validados por la autora permitiendo incluso, en algunos casos, realizar pequeñas aportaciones cuando los detalles de implementación de los \textit{papers} originales no son lo suficientemente explícitos (consultar sección~\ref{s-experimentacion-w-cof}).


\section{Estructura de la memoria}

El proyecto ha sido dividido en dos documentos independientes pero complementarios.

\subsection{Memoria}

Documento principal donde se pueden encontrar los siguientes apartados:

\begin{enumerate}
	\item \textbf{Introducción}: contextualización, estructura y recursos del proyecto.
	\item \textbf{Objetivos}: exposición de los pilares básicos del trabajo y las metas propuestas.
	\item \textbf{Conceptos teóricos}: descripción de los fundamentos principales necesarios para la correcta comprensión del documento.
	\item \textbf{Técnicas y herramientas}: enumeración de algunas de las técnicas y herramientas utilizadas durante el desarrollo del proyecto.
	\item \textbf{Aspectos relevantes del proyecto}: exposición y análisis de los experimentos realizados y las aproximaciones propuestas.
	\item \textbf{Trabajos relacionados}: comparación entre literatura existente desarrollada en el mismo ámbito que el trabajo presentado.
	\item \textbf{Conclusiones y líneas de trabajo futuras}: resumen de los aprendizajes obtenidos y proposición de ideas de mejora.
\end{enumerate}

\subsection{Anexos}

Documento orientado hacia el desarrollo del producto entregado, se divide en las siguientes secciones:

\begin{enumerate}
	\item \textbf{Plan de proyecto \textit{software}}: discusión acerca de la viabilidad temporal, económica y legal del producto presentado.
	\item \textbf{Especificación de requisitos}: análisis de las características que debe poseer el proyecto \textit{software} mediante el uso de catálogos, prototipos, diagramas y tablas.
	\item \textbf{Especificación de diseño}: descripción detallada de la estructura,
	el comportamiento y la interacción de la propuesta entregada.
	\item \textbf{Documentación técnica de programación}: recursos diseñados para disminuir la curva de aprendizaje y facilitar la inclusión de otros desarrolladores en el proyecto.
	\item \textbf{Documentación de usuario}: manuales necesarios para garantizar que el usuario final del producto pueda utilizar el \textit{software} con facilidad y reciba soporte adecuado.
\end{enumerate} 

\section{Recursos adicionales}
\label{s:recursos}
Se facilita a continuación el acceso directo a los materiales del proyecto.

\subsection{Repositorio principal}

El repositorio donde se encuentra almacenado el código fuente, la documentación, los \textit{scripts} de despliegue y algunos materiales adicionales (como fondos de pantalla) se encuentra alojado en GitHub.

\begin{itemize}
	\item \textbf{Repositorio del proyecto:}
	\url{https://github.com/phf1001/semisupervised-learning-in-cibersecurity}
\end{itemize}


\subsection{Documentación y ayuda}

Además de los documentos estándar (memoria y anexos), se ha realizado una \textit{wiki} de GitHub donde se explican las principales funcionalidades y características de la \textit{web} desarrollada. Como complemento, se ha creado un canal de YouTube con tutoriales detallados de funcionamiento, despliegue y otras características.

\begin{itemize}
	\item \textbf{\textit{Wiki} del proyecto:} 	\url{https://github.com/phf1001/semisupervised-learning-in-cibersecurity/wiki}
	\item \textbf{Canal de YouTube con tutoriales de uso y despliegue:} \url{https://www.youtube.com/@KRINIPHISHINGSCANNER/playlists}
\end{itemize}


\subsection{\textit{Web} desarrollada}

Como se explicará más adelante, utilizando como base el conocimiento adquirido durante la fase de investigación, se ha desarrollado un analizador de \textit{phishing} que se encuentra disponible en Heroku. Sin embargo, en algunos casos, el proceso de análisis puede requerir demasiado tiempo para esta plataforma. Por ello, se facilitan imágenes de Docker con las que se puede desplegar la \textit{web} mediante un contenedor en local. El proceso está automatizado mediante \textit{scripts} disponibles para diversos sistemas operativos.

\begin{itemize}
	\item \textbf{\textit{Web} desplegada en Heroku:} \url{https://krini.herokuapp.com/}
	\begin{itemize}
		\item Cada miembro del tribunal posee un usuario administrador para garantizar el acceso a la funcionalidad completa de la aplicación. Las \textbf{credenciales} se encuentran en un fichero de texto disponible en las \textbf{memorias USB} entregadas en secretaría.
	\end{itemize}
	\item \textbf{Imágenes de Docker con la \textit{web} y su base de datos:} \url{https://hub.docker.com/r/phf1001/krini/tags}
	\begin{itemize}
		\item \textbf{\textit{Scripts} de despliegue para Docker en distintos sistemas operativos:} las instrucciones de uso se encuentran disponibles en el \texttt{README.md} del directorio, junto con enlaces a videotutoriales. 
		
		Disponibles en: \url{https://github.com/phf1001/semisupervised-learning-in-cibersecurity/tree/main/docker-deploy-kit}
	\end{itemize}

\end{itemize}