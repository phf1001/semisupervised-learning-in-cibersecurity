\capitulo{1}{Introducción}

[Descripción del contenido del trabajo y del estrucutra de la memoria y del resto de materiales entregados..]


\section{Preámbulo}

A diferencia de unas décadas atrás, la sociedad actual está gobernada por los datos. La transición a la época de la información puede ser compleja para determinados colectivos y, consecuentemente, diversos sistemas auxiliares han sido desarrollados con el fin de resumir información y facilitar la toma de decisiones. Entre ellos se encuentran los sistemas de recomendación, que son herramientas que pretenden realizar proposiciones de objetos que pueden resultar interesantes para un determinado perfil.

Económicamente, este tipo de algoritmo es un claro objeto de interés, puesto que puede influir en la toma de decisiones de los compradores y hacer que se inclinen por un determinado producto (por ejemplo, el que tenga una mejor valoración). Los atacantes conocen esta situación y manipulan estas herramientas mediante el uso de perfiles falsos con el fin de beneficiar sus productos o perjudicar los de la competencia.

Este proyecto de investigación pretende proponer un algoritmo de Aprendizaje Semisupervisado que sea capaz de detectar los ataques a sistemas de recomendación, diferenciando entre perfiles genuinos e inyectados, además de comprobar la veracidad de los planteados por otros investigadores.


