\capitulo{1}{Introducción}

A diferencia de unas décadas atrás, la sociedad actual está gobernada por los datos. La transición a la era de la información puede ser compleja para determinados colectivos y, consecuentemente, han surgido grupos que pretenden facilitar la experiencia de los usuarios, mientras que otros buscan aprovechar la falta de información general para lograr un beneficio.

Este trabajo de investigación pretende explorar cómo el aprendizaje semisupervisado puede aplicarse a la seguridad informática con el fin de lograr un entorno digital de confianza. Para ello, se han abierto dos ramas de investigación: la detección automática de ataques en sistemas de recomendación y \textit{phishing.}

Un sistema de recomendación es un algoritmo desarrollado con el fin de resumir información y facilitar la toma de decisiones. Son herramientas que pretenden realizar sugerencias de objetos que pueden resultar interesantes para un determinado perfil. Económicamente, este tipo de algoritmo es un claro objeto de interés, puesto que puede influir en la toma de decisiones de los compradores y hacer que se inclinen por un determinado producto (por ejemplo, el que tenga una mejor valoración). Los atacantes conocen esta situación y manipulan estas herramientas mediante el uso de perfiles falsos con el fin de beneficiar sus productos o perjudicar los de la competencia. Por ello, uno de los objetivos del proyecto es explorar cómo el aprendizaje semisupervisado puede ayudar a detectar los ataques a sistemas de recomendación, diferenciando entre perfiles genuinos e inyectados, además de comprobar la veracidad de los trabajos previos planteados por otros investigadores.

Por otro lado, se denomina \textit{phishing} a la técnica de ingeniería social en la que un atacante suplanta la identidad de una entidad de confianza con el fin de extraer datos o información personal de un determinado usuario. Este tipo de ataque es uno de los más comunes y peligrosos. Según el informe \textit{State of the Phish Report}, en el 2021, un 83\% de las empresas encuestadas sufrieron un ataque de estas características y el 54\% acabó con brechas de datos de clientes~\cite{phishingPorcentajeExito}. Por ello, se pretende crear un sistema de detección que permita distinguir enlaces falsos de auténticos mediante ténicas de aprendizaje semisupervisado.