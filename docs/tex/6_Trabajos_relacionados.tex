\capitulo{6}{Trabajos relacionados}

Dentro de este proyecto se pueden diferenciar distintas líneas de investigación, principalmente las dirigidas hacia el desarrollo y comprensión de los algoritmos de aprendizaje semisupervisado y las centradas en formalizar ataques a sistemas de recomendación.

\subsection{Aprendizaje semisupervisado aplicado a la detección de ataques en sistemas de recomendación}

\subsubsection{\textit{Co-Forest} aplicado a la detección de ataques~\cite{zhou2021SemisupervisedRecommendationAttack}}
En esta sección, el artículo fundamental es <<\textit{Semi-supervised recommendation attack detection based on Co-Forest}>>~\cite{zhou2021SemisupervisedRecommendationAttack}. En este \textit{paper}, se propone un método de detección basado en Co-Forest y se producen distintas comparativas con otros algoritmos para comprobar su eficacia, consiguiendo unos resultados muy aceptables. Partiendo de esta base nace el presente documento, que pretende explorar la solución propuesta por estos autores y expandirla.

\subsubsection{\textit{Naive Bayes} aplicado a la detección de ataques~\cite{zhiang2012HySADNayveBayes}}
También es muy relevante citar el trabajo expuesto en <<\textit{HySAD: A semi-supervised hybrid shilling attack detector for trustworthy product recommendation}>>~\cite{zhiang2012HySADNayveBayes}, puesto que propone una aproximación Naive Bayes para separar perfiles de atacantes de perfiles genuinos y además utiliza los tipos de \textit{datasets} que son probados posteriormente por Zhou y Duan (Amazon, Netflix y MovieLens). Se trata de uno de los trabajos de referencia en el área.

\subsection{Ataques en sistemas de recomendación}

La importancia de proteger los sistemas de recomendación ha sido contemplada desde principio de siglo, siendo común la proposición de otros tipos de aprendizaje para detectar los ataques.

\subsubsection{Recolección de los tipos de ataques y propuestas de reconocimiento~\cite{mingdan2018ShillingAttacksAReview}}
Respecto a la descripción de los tipos de intrusión, la correcta definición formal (matemática) de sus parámetros y una recopilación de la gran mayoría de ataques existentes, es fundamental referenciar el artículo de Mingdan, Quingshan <<\textit{Shilling attacks against collaborative recommender systems: a review}>>~\cite{mingdan2018ShillingAttacksAReview}. Se trata de una recopilación reciente (2018) de las principales investigaciones de los autores más populares en la materia que destaca por su completitud.

\subsubsection{Definición de conceptos~\cite{Mobasher2006Thesis}}
Previo a este documento,también es relevante contemplar otros trabajos, como la tesis de William y Mobasher <<\textit{Thesis: Profile injection attack detection for securing collaborative recommender systems}>>~\cite{Mobasher2006Thesis}. En ella se introduce el concepto de inyección y se parametrizan características como el tamaño de ataque. Es destacable la autoridad de estos investigadores en la materia, siendo propietarios de muchos documentos de interés.

	\subsubsection{Primeras definiciones formales~\citep{mahony2004CollaborativeRecommendation}}
<<\textit{Collaborative recommendation: A robustness analysis}>>~\cite{mahony2004CollaborativeRecommendation}, de O'Mahony y Hurley, es uno de los trabajos con más antiguedad pero mayor número de referencias que se encuentra. En él se definen los modelos de construcciones en base a conocimiento del sistema y pone a prueba la robustez de los recomendadores evaluando su estabilidad y precisión ante la presencia de perfiles inyectados (análisis matemático muy completo).

\subsection{Algoritmos de aprendizaje semisupervisado}

\subsubsection{\textit{Co-Forest}}

A pesar de que el paper principal del proyecto (Zhou y Duan~\cite{zhou2021SemisupervisedRecommendationAttack}) expone que se utiliza la versión original del \textit{co-forest}~\cite{originalCoForest2007}, es destacable mencionar que ciertos aspectos de implementación no han sido contemplados en estos artículos. Por ello, la implementación propia se ha basado también en <<\textit{Semi-supervised ensemble learning. Master’s thesis}>>, desarrollada por Engelen y Hoos~\cite{engelen2018thesis}.