\capitulo{6}{Trabajos relacionados}

Dentro de este proyecto se pueden diferenciar tres líneas de investigación:


\section{Aprendizaje semisupervisado}

Como principal referente, se ha utilizado el artículo de Jesper E. van Engelen y Holger H. Hoos \citep{engelen-hoos}. En él, se expone una clasificación general de los principales tipos de métodos derivados del aprendizaje semisupervisado y sus características. Auxiliarmente, también ha sido consultado el documento de Triguero, García y Herrera \citep{triguero-garcia-herrera}. 


\section{Aprendizaje semisupervisado aplicado a la detección de ataques en sistemas de recomendación}

En este caso, el artículo fundamental es \citep{zhou-duan}. En este documento, se propone un método de detección basado en Co-Forest y se producen distintas comparativas con otros algoritmos para comprobar su eficacia, consiguiendo unos resultados muy aceptables. También es muy relevante citar el trabajo de \citep{zhiang-junjie}, puesto que propone una aproximación Naive Bayes para separar perfiles de atacantes de perfiles genuinos y además propone los datasets que son utilizados posteriormente por Zhou y Duan (Amazon, Netflix y MovieLens).

\section{Ataques en sistemas de recomendación}

La importancia de proteger los sistemas de recomendación ha sido contemplada desde principio de siglo, siendo común la proposición de otros tipos de aprendizaje para detectar los ataques.

Respecto a la descripción de los tipos de intrusión, la correcta definición formal (matemática) de sus parámetros y una recopilación de la gran mayoría de ataques existentes, es fundamental referenciar el artículo de \citep{mingdan-qingshan}. Previo a este documento,también es relevante contemplar otros trabajos, como la conferencia de \citep{lam-riedl}, donde se propone utilizar como datasets los conjuntos de películas o el paper de \citep{mahony}, que define los modelos de construcciones en base a conocimiento del sistema y pone a prueba la robustez de los recomendadores evaluando su estabilidad y precisión ante la presencia de perfiles inyectados (análisis matemático muy completo).